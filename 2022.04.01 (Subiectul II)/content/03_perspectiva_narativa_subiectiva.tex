\section{Perspectiva narativă subiectivă}

Perspectiva narativă este punctul de vedere al naratorului, unghiul din care privește și interpretează faptele.

În fragmentul supus analizei, extras din opera literară \ex{titlul} de \ex{autorul}, perspectiva narativă este subiectivă, cu focalizare internă și viziune „împreună cu”.

Subiectivitatea este evidențiată prin folosirea formelor verbale și pronominale de persoana \rom{1} (\ex{3 verbe, 3 pronume}), realizată din unghiul unui narator homodiegetic (subiectiv) ce povestește din postura personajului implicat în acțiune.

Focalizarea internă accentuează gândurile și sentimentele naratorului personaj ce evocă momentul în care \ex{ex. din text + citat}. Este evidentă \ex{ex: evoluția, emoția} naratorului personaj, a cărui autocaracterizare ilustrează \ex{exemplu}.

Având caracter subiectiv, acțiunea este prezentată într-o ordine personală, dictată de propria conștiință, naratorul concentrându-se asupra \ex{ex: universul său sufletesc}. În acest sens, se accentuează \ex{ex: conflictul interior, ...}

Astfel, subiectivitatea perspectivei narative conferă un caracter psihologic, afectiv evenimentelor relatate.
