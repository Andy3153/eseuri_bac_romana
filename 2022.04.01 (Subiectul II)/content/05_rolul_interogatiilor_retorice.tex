\section{Rolul interogațiilor retorice}

Interogația, folosită ca mijloc retoric, face mai percutant contactul dintre mesaj și receptor, posedând o facultate deosebită de a contribui la realizarea unui plus de cunoaștere.

În fragmentul supus analizei, extras din opera literară \ex{titlul} de \ex{autorul}, interogațiile retorice au un dublu rol. Pe de o parte, \ex{ex: personajul prezintă pulicului Parisul („ce} \ex{nu găsești în el?”)}

Pe de altă parte, interogațiile retorice contribuie la caracterizarea indirectă a personajului. \ex{ex: astfel, se creează un efect comic de limbaj, legat de caracterul par-} \ex{venit al cucoanei ...}

Așadar, interogațiile retorice se dovedesc a fi un procedeu artistic, prin intermediul căruia se exprimă sentimente și gânduri complexe.
