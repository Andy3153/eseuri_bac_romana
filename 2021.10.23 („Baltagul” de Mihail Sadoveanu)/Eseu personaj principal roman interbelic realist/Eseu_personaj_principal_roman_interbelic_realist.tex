% Include the preamble
\input{../../preamble.tex}

\newcommand{\operatitle}{\textbfit{„Baltagul”}} % title of the text
\newcommand{\operaauthor}{Mihail Sadoveanu} % author of the text

\title{Particularități de construcție a personajului principal dintr-un roman interbelic, realist}


\begin{document}
\maketitle % Show the title
%\reversemarginpar % put margin notes on left instead of on right

% Beginning of text


\subsection{Context}

Opera \operatitle\ de \operaauthor, publicată în 1930, este un roman interbelic, obiectiv, realist-mitic și tradițional.

\marginnote{aspectul realist}[0.3cm]
Romanul este realizat pe două coordonate fundamentale: aspectul realist (monografia lumii pastorale, reperele spațiale, tipologia personajelor, tehnica detaliului semnificativ) și aspectul mitic (gesturile rituale ale Vitoriei, tradițiile pastorale, motivul comuniunii om-natură și mitul marii treceri).

\marginnote{tema}[0.8cm]
Tema rurală a romanului tradițional este dublată de tema călătoriei inițiatice și justițiare. Romanul \operatitle\ prezintă monografia satului moldovenesc de la munte, lumea arhaică a păstorilor, având în prim-plan căutarea și pedepsirea celor care l-au ucis pe Nechifor Lipan. Însoțită de Gheorghiță, Vitoria reconstituie drumul parcurs de bărbatul său, pentru descoperirea adevărului și înfăptuirea dreptății.


\section{Ilustrarea elementelor de structură și de compoziție ale romanului, semnificative pentru realizarea personajului din romanul studiat {\footnotesize\normalfont (de exemplu: acțiune, conflict, relații temporale și spațiale, incipit, final, tehnici narative, perspectivă narativă, registre stilistice, limbajul personajelor etc.)}}

\marginnote{titlul}[0.7cm]
Titlul \operatitle\ înseamnă toporul cu două tăișuri, un obiect simbolic, ambivalent: armă a crimei și instrument al dreptății. De remarcat că în roman, același baltag (al lui Lipan) îndeplinește cele două funcții. Baltagul tânărului Gheorghiță se păstrează neatins de sângele ucigașilor.

\marginnote{narațiunea}[0.3cm]
Narațiunea se face la persoana a \rom{3}-a, iar naratorul omniprezent și omniscient reconstituie lumea satului de munteni și acțiunile Vitoriei, în mod obiectiv, prin tehnica detaliului și observație.

\marginnote{timpul}[0.3cm]
Timpul derulării acțiunii este vag precizat, prin repere temporale din calendarul religios al satului tradițional: \textit{„aproape de Sf. Andrei”}, \textit{„în Postul Mare”}, \textit{„10 Martie”}. Cadrul acțiunii este satul de munte Măgura Tarcăului, zona Dornelor și a Bistriței, dar și satul de câmpie, Cristești, în Baia Jijiei.

Romanul este structurat în șaisprezece capitole cu acțiune desfășurată cronologic, urmărind momentele subiectului. În raport cu tema călătoriei, capitolele pot fi grupate în trei părți: \rom{1}. constatarea absenței și pregătirile de drum, \rom{2}. căutarea soțului dispărut, \rom{3}. găsirea celui căutat, înmormântarea și pedepsirea făptașilor.

\marginnote{incipitul}[0.3cm]
Incipitul romanului este o legendă despre ocupațiile și modul de viață al păstorilor și al altor neamuri, pe care o spunea Nechifor la \textit{„cumătrii și nunți”}. Legenda este rememorată de Vitoria în absența soțului ei și anticipează destinul acestuia, având rol de prolog. Finalul (epilogul) cuprinde planurile de viitor ale Vitoriei în legătură cu familia ei, rostite după încheierea deznodământului.


\section{Precizarea statutului social, psihologic, moral etc. al personajului ales}

Personajul principal este Vitoria Lipan, femeia voluntară, munteancă, soție de cioban și mamă autoritară. Vitoria este o femeie puternică, hotărâtă să-și găsească soțul, să facă dreptate și să-l înmormânteze creștinește.

Este o femeie conservatoare, respectă obiceiurile, tradițiile satului patriarhal și respinge noutățile aduse de copii.


\section{Ilustrarea trăsăturilor personajului ales, prin secvențe narative/situații semnificative sau prin citate comentate}

Vitoria reprezintă tipul femeii voluntare, fiind \textit{„un exponent al speței”} în raport cu lumea arhaică, dar și o individualitate prin însușirile sale: spiritul de răzbunare și metodele unui detectiv. Criticul G. Călinescu afirmă \textit{„Vitoria e un Hamlet feminin”}, pentru că pune la cale demascarea ucigașilor la parastas, când reconstituie scena morții lui Nechifor Lipan și o povestește în fața sătenilor.

Este o femeie hotărâtă, curajoasă, lucidă. Pe drum își ia o pușcă, pe care nu ezită să o folosească, iar lui Gheorghiță îi dă un baltag sfințit. Are curajul de a depăși hotarul satului, în căutarea lui Nechifor, motiv pentru care criticul Nicolae Manolescu o numește \textit{„o femeie în țara bărbaților”}. Datorită inteligenței native și a stăpânirii de sine, Vitoria reușește să reconstituie drumul parcurs de Lipan, să afle adevărul și să demaște ucigașii în fața autorităților.

Găsește în sine puterea de a cerceta și de a găsi urmele acestuia, cu inteligență și disimulare.

Vitoria transmite copiilor respectul tradițiilor și este refractară la noutățile civilizației: \textit{„În tren ești olog, mut și chior”}, îi spune lui Gheorghiță. Ca mamă, îi interzice Minodorei să se îndepărteze de tradiție (\textit{„Îți arăt eu coc, valț și bluză...! Nici eu, nici bunică-ta, nici bunică-mea n-am știut de acestea -- și-n legea noastră trebuie să trăiești și tu!”}) și contribuie prin călătorie la maturizarea lui Gheorghiță. În raport cu Gheorghiță, tânărul imatur la începutul călătoriei, Vitoria este inițiatoarea. Deși fiul este copilul preferat și poartă numele secret al tatălui, el constată pe drum că mamei \textit{„i-au crescut țepi de aricioaică”}, adică a devenit aspră cu el. Inițierea se continuă cu proba coborârii în râpă (amintind de mitul coborârii în Infern) și a vegherii nocturne la rămășițele tatălui, probă a curajului bărbătesc, la ordinul repetat al mamei: \textit{„-Coboară-te în râpă îți spun!”}. Inițierea este încheiată la parastas, când tânărul ține piept criminalului și-l lovește cu baltagul, restabilind dreptatea.

Vitoria respectă obiceiurile de cumetrie și de nuntă și veghează la îndeplinirea rânduielilor din ritualul înmormântării: priveghiul, drumul la cimitir, bocitul, slujba religioasă, pomana, praznicul. Știe să citească semnele naturii: la Dorna, la Crucea Talienilor, vântul o anunță că se află pe drumul cel bun, iar la moș Pricop, în Fărcașa, ceața îi dă semn de popas.

Soție iubitoare, pornește hotărâtă în căutarea bărbatului. Țipătul dinaintea coborârii coșciugului și gesturile concentrează iubirea și durerea pierderii soțului.

Personajul complex este realizat prin tehnica basoreliefului și individualizat prin caracterizare directă și indirectă (prin fapte, vorbe, atitudini, gesturi, relații cu alte personaje, nume).

Portretul fizic relevă frumusețea personajului prin tehnica detaliului semnificativ: \textit{„Nu mai era tânără, dar avea o frumuseță neobișnuită în privire”}.


\subsection{Concluzie}

În concluzie, personajul Vitoria Lipan din opera litarară \operatitle\ scrisă de \operaauthor\ este un personaj complex.
\end{document}
