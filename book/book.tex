%%
%% Book containing everything in this repo
%% COMPILE WITH XeLaTeX
%% IN TESTING CURRENTLY
%%

\documentclass[
 12pt,                       % Font size
 a4paper                     % Paper type
]{book}

% Packages
\usepackage[
 margin=2.7cm,               % Margin size
 marginparwidth=2cm,         % Margin note size
 marginparsep=3mm,           % Space between margin and text
 headheight=15pt             % Header height (fix that stupid warn from fancyhdr)
]{geometry}
\usepackage[romanian]{babel} % Romanian characters support
\usepackage{indentfirst}     % Add paragraph indentation even after a section
\usepackage{marginnote}      % Notes on the margins of a document (more advanced \marginpar)
\usepackage{titlesec}        % Customize titles
\usepackage{hyperref}        % Hyperlink support
\usepackage{graphicx}        % Image support
\usepackage{xcolor}          % Custom colors
\usepackage{fancyhdr}        % Custom headers
\usepackage{etoolbox}        % Customize chapters
\usepackage[titles]{tocloft} % Customize ToC
\usepackage{blindtext}       % Sample filler text

% Image path
\graphicspath{ {./src/} }

% Hyperlink configuration
\hypersetup{
 colorlinks=true,
 urlcolor=black,
 linkcolor=black  % ToC links color
}

% Colors
\definecolor{gray75}{gray}{0.75}

% Custom format for titles, sections, subsecions etc.
\titleformat{\chapter}[hang]{\Large}{\thechapter\hspace{15pt}\textcolor{gray75}{|}\hspace{15pt}}{0pt}{\Large}
\renewcommand{\thesection}{\arabic{section}}%
\titleformat*{\section}{\large\bfseries}
\titleformat{\subsection}{\normalfont\normalfont\bfseries}{}{1.5em}{}

% Custom commands
% Format: \newcommand{\command}[variable]{action #variable}
\newcommand{\rom}[1]{\uppercase\expandafter{\romannumeral #1\relax}} % Roman numerals
\newcommand{\textbfit}[1]{\textbf{\textit{#1}}}                      % combine bold and italic
\newcommand{\operatitle}{}                                           % to not get errors
\newcommand{\operaauthor}{}                                          % to not get errors

% Customize \marginnote font
\renewcommand\marginfont{\ttfamily\footnotesize}

% Make \ttfamily hyphenate words for the margin notes
\DeclareFontFamily{OT1}{cmtt}{\hyphenchar\font=-1}
\DeclareFontFamily{\encodingdefault}{\ttdefault}{\hyphenchar\font=`\-}
\DeclareFontFamily{T1}{cmtt}{\hyphenchar\font=45}

% ToC customization
\setcounter{tocdepth}{0}     % Make only chapters appear in ToC
\renewcommand{\cftdot}{}     % Remove ToC dots

% Custom fancy styles
\fancypagestyle{chapterfancystyle}{
 \fancyhf{}
 \fancyfoot[LE,RO]{\thepage}
 \renewcommand{\headrulewidth}{0pt}
 \renewcommand{\footrulewidth}{1pt}
}

\fancypagestyle{tocfancystyle}{
 \fancyhf{}
 \fancyhead[LE]{Cuprins}
 \fancyfoot[LE,RO]{\thepage}
 \renewcommand{\headrulewidth}{2pt}
 \renewcommand{\footrulewidth}{1pt}
}

\fancypagestyle{plain}{
 \fancyhf{}
 \fancyhead[LE]{\small\textsc{Esee pentru bacalaureat}}
 \fancyhead[RO]{
  \begingroup
    \small
    \let\textbfit\relax
    \textsc{\operatitle\ -- \operaauthor}
  \endgroup
 }
 \fancyfoot[LE,RO]{\thepage}
 \renewcommand{\headrulewidth}{2pt}
 \renewcommand{\footrulewidth}{1pt}
}

% Patch chapter to use custom header
\patchcmd{\chapter}{\thispagestyle{plain}}{\thispagestyle{chapterfancystyle}}{}{}



%%%%%%%%%%%%%%%%%%



\begin{document}
\pagestyle{empty}
% Create a title page
\begin{titlepage}
 \centering
 \vspace*{1cm}
 \vspace{4\baselineskip}
 {\Huge
 Esee pentru bacalaureatul la \\ Limba și Literatura Română\par}
 \vspace{4\baselineskip}
 de\par
 {\Large\textsc{Dobrete Andrei-Robert}\par}
 \vfill
 \includegraphics[height=\fontcharht\font`\B]{gitlab}
 \url{gitlab.com/Andy3153/esee_bac_romana}\par
 \includegraphics[height=\fontcharht\font`\B]{github}
 \url{github.com/Andy3153/esee_bac_romana}\par
 \vspace{1.5\baselineskip}
 {\large\LaTeXe}
\end{titlepage}

% Disable bold in ToC
\addtocontents{toc}{\string\renewcommand{\protect\cftchappagefont}{\protect\normalfont}}
\addtocontents{toc}{\string\renewcommand{\protect\cftchapfont}{\protect\normalfont}}
\addtocontents{toc}{\string\renewcommand{\protect\cftchapleader}{\protect\normalfont\protect\cftdotfill{\protect\cftsecdotsep}}}


\tableofcontents % Generate the ToC

%\addtocontents{toc}{~\hfill\textbf{Pagina}\par} % text above page nr

\pagestyle{plain}


% Beginning of text

% esee
% harap-alb
\chapter{Eseu cu privire la tema și viziunea despre lume dintr-un basm cult studiat}
% Commands
\renewcommand{\operatitle}{\textbfit{„Riga Crypto și lapona Enigel”}} % title of the text
\renewcommand{\operaauthor}{Ion Barbu} % author of the text


% Beginning of text
\subsection{Context}

Publicată în 1924, integrată apoi în volumul \textbfit{„Joc secund”}, balada \operatitle\ face parte din a doua etapă de creație barbiană, numită baladic-orientală, dar anunță dezvoltarea ulterioară a poeziei lui \operaauthor.


\section{Evidențierea trăsăturilor care fac posibilă încadrarea poeziei studiate într-o tipologie, într-un curent cultural/literar, într-o orientare tematică}

\marginnote{poem alegoric}[1.2cm]
\operatitle\ este subintitulată \textit{„Baladă”}, începe ca un cântec bătrânesc de nuntă, dar se realizează în viziune modernă, ca un amplu poem de cunoaștere și poem alegoric, o poveste de iubire din lumea vegetală. Autorul păstrează din specia tradițională schema epică și personajele antagonice, dar evenimentele narate sunt de natură fantastică (dialogul în vis dintre rigă și laponă) și alegorică. Scenariul epic este dublat de caracterul dramatic și de \textit{„lirismul de măști”}, personajele având semnificații simbolice multiple (materia și spiritul etc.).

Poemul se încadrează modernismului interbelic prin intelectualizarea emoției, imaginar poetic inedit, ambiguitate, metafore surprinzătoare și cuvinte cu sonorități neobișnuite, înnoiri prozodice.


\section{Prezentarea imaginilor/ideilor poetice, relevante pentru te\-ma și viziunea despre lume din textul studiat}

Tema poeziei o reprezintă iubirea ca modalitate de cunoaștere a lumii. Fiind „un \textbfit{Luceafăr} întors”, poemul prezintă drama cunoașterii și a incompatibilității dintre două lumi (regnuri).

Titlul baladei trimite cu gândul la marile povești de dragoste din literatura universală, \textbfit{„Romeo și Julieta”}, \textbfit{„Tristan și Isolda”}. Însă la \operaauthor, membrii cuplului sunt antagonici (fac parte din regnuri diferite). Sunt personaje romantice cu trăsături excepționale, dar negative în raport cu norma comună (Crypto e \textit{„sterp”} și \textit{„nărăvaș/Că nu voia să înflorească”}, iar Enigel este \textit{„prea-cuminte”}).

\marginnote{semnificația titlului}[-0.3cm]
Numele Crypto are dublă semnificație: cel tăinuit, \textit{„inimă ascunsă”}, provenind din adjectivul \textit{„criptic”}, (\textit{„ascuns”}, \textit{„tăinuit”}), dar sugerează, în egală măsură, apartenența sa la familia ciupercilor, numele științific \textit{„criptogame”}. Personajul este rege (rigă) al făpturilor inferioare, din regnul vegetal. Numele cu sonoritate nordică Enigel sugerează originea laponei (de la pol) și trimite probabil la semnificația cuvântului din limba suedeză, \textit{„înger”} (care provine din latinescul \textit{„angelus”}).


\section{Ilustrarea elementelor de compoziție și de limbaj ale textului poetic studiat, semnificative pentru tema și viziunea despre lume {\footnotesize\normalfont(de exemplu: imaginar poetic, titlu, incipit, relații de opoziție și de simetrie, motiv poetic, laitmotiv, figuri semantice/tropi, elemente de prozodie etc.)}}

\marginnote{compoziție}[0.3cm]
La nivel formal, poezia este alcătuită din două părți, fiecare dintre ele prezentând câte o nuntă: una împlinită, cadru al celeilalte nunți, povestită, ratată, modificată în final prin căsătoria lui Crypto cu măsălarița. Formula compozițională este aceea a povestirii în ramă.

Prologul conturează în puține imagini atmosfera de la finalul unei nunți trăite. Primele patru strofe constituie rama viitoarei povești și reprezintă dialogul menestrelului cu \textit{„nuntașul fruntaș”}.

Partea a doua prezintă povestea de iubire neîmplinită dintre Enigel și riga Crypto. Nunta povestită cuprinde mai multe tablouri poetice: portretul și împărăția rigăi Crypto (strofele 5 -- 7), portretul, locurile natale și oprirea din drum a laponei Enigel (strofele 8 -- 9), întâlnirea dintre cei doi (strofa 10), cele trei chemări ale rigăi și primele două refuzuri ale laponei (strofele 11 -- 15), răspunsul laponei și refuzul categoric cu relevarea relației dintre simbolul solar și propria condiție (strofele 16 -- 20), încheierea întâlnirii (strofele 21 -- 22), pedepsirea rigăi în finalul baladei (strofele 23 -- 27). Modurile de expunere sunt, în ordine: descrierea, dialogul și narațiunea.

În expozițiune, sunt prezentate în antiteză portretele membrilor cuplului și locurile lor natale, deosebirile dintre ei generând intriga.

Riga Crypto, \textit{„inimă ascunsă”}, este craiul bureților, căruia dragostea pentru Enigel, \textit{„laponă mică, liniștită”}, îi este fatală. Singura lor asemănare este statutul superior în interiorul propriei lumi: el este rigă al plantelor inferioare, care nu înfloresc, iar păstorița care își conduce turmele de reni spre sud este o stăpână a regnului animal, în ipostaza de ființă rațională, omul -- \textit{„fiară bătrână”}.

\marginnote{imaginar poetic}[0.3cm]
Spațiul definitoriu al existenței, pentru Crypto, este umezeala perpetuă și impură, în timp ce lapona vine \textit{„din țări de gheață urgisită”}, spațiu rece, ceea ce explică aspirația ei spre soare și lumină, dar și mișcarea de transhumanță care ocazionează popasul în ținutul rigăi.

Membrii cuplului fac parte din regnuri diferite și, de aceea, nu pot comunica în plan real. Întâlnirea lor se realizează în visul fetei, la fel ca în \textbfit{„Luceafărul”}. Riga este cel care rostește de trei ori descântecul de dragoste și, de tot atâtea ori, lapona îl respinge. Povestea propriu-zisă se dovedește a fi fantastică, ca și în poemul eminescian, doar că rolurile sunt inversate. În dialogul lor, formulele de adresare sugerează familiaritate, afecțiune blândă: repetiția \textit{„Enigel, Enigel”}, epitetul \textit{„rigă blând”}.

În prima chemare-descântec, cu rezonanțe de incantație magică, Crypto își îmbie aleasa cu \textit{„dulceață”} și cu \textit{„fulgi”}, elemente ale existenței sale vegetative, dar care aici capătă conotații erotice. Darul lui este refuzat categoric de Enigel: \textit{„Eu mă duc să culeg/Fragii fragezi mai la vale”}. Refuzul laponei îl pune într-o situație dilematică, dar opțiunea lui e fermă și merge până la sacrificiul de sine, în a doua chemare: \textit{„Dacă pleci să culegi/Începi, rogu-te, cu mine”}.

Al doilea refuz este susținut de enumerarea atributelor lui Crypto: \textit{„blând”}, \textit{„plăpând”}, \textit{necopt -- „Lasă. Așteaptă de te coace”}. Opoziția \textit{„copt”} -- \textit{„necopt”}, reluată în al treilea refuz prin antiteza \textit{soare-umbră}, pune în evidență incompatibilitatea lor. Imaginii de fragilitate a lui Crypto lapona îi opune aspirația ei spre absolut. Soarele este simbolul existenței spirituale, al împlinirii umane, în antiteză cu \textit{„umbra”}, simbol al existenței instinctuale, sterile, vegetative.

Pentru a-și continua drumul către soare și cunoaștere, lapona refuză descântecul rigăi, deși regretă și plânge. Descântecul se întoarce în mod brutal asupra celui care l-a rostit și-l distruge. Făptura firavă este distrusă de propriul vis, cade victimă neputinței și îndrăznelii de a-și depăși limitele.

Finalul este trist. Riga Crypto se transformă într-o ciupercă otrăvitoare, obligat să rămână alături de făpturi asemenea lui, \textit{„Laurul-Balaurul”} și \textit{„măsălarița-mireasă”}. Încercarea ființei inferioare de a-și depăși limitele este pedepsită cu nebunia.

\marginnote{figuri semantice/ tropi}[1.3cm]
Soarele, simbolul spiritului, este imaginat în poem prin metaforele \textit{„roata albă”} (perfecțiunea geometrică) și \textit{„aprins inel”} (simbolul nunții), în antiteză cu \textit{„umbra”}, iar metafora \textit{„sufletul-fântână”} sugerează puritatea, setea de cunoaștere, veșnicia, fiind în antiteză cu \textit{„carnea”} (trupul, instinctele). Spiritul și sufletul sunt atribute ale ființei raționale, înțelepte. Făpturile inferioare care aspiră să dobândească spiritualitate sunt distruse de propriul vis, așa cum i se întâmplă lui Crypto, care „înnebunește” și se transformă în ciupercă otrăvitoare.

Trei mituri fundamentale de origine greacă sunt valorificate în opera poetului: al soarelui (absolutul), al nunții și al oglinzii.

Drumul spre sud al laponei are semnificația unui drum inițiatic, iar popasul în ținutul rigăi este o probă, trecută prin respingerea nunții pe o treaptă inferioară.
\marginnote{semnificații}[0.3cm]
Drumul trece prin cercul Venerii (iubirea ce reduce omul la ipostază de ființă instinctuală), apoi sufletul trebuie să mai urce o treaptă, cercul lui Mercur, mai pur, al intelectului, al cunoașterii raționale. Inițierea completă are loc prin adevărata \textit{nuntă} a trupului și spiritului cu însuși focarul vieții, prin trecerea omului în cercul Soarelui (cunoașterea absolută). Aspirația solară a laponei sugerează faptul că, în momentul întâlnirii cu riga Crypto, aceasta se află pe treapta lui Mercur, fără ca ea să fi trăit experiența iubirii. Chemările lui Crypto o atrag spre cercul Venerii. Ea trăiește iubirea ca experiență inițiatică, dar alege să-și urmeze drumul spre Soare (cunoașterea absolută).

\marginnote{figuri de stil}[0.8cm]
Sub raport stilistic, prezența inversiunilor (\textit{„zice-l-aș”}) și a vocativelor în prima parte a baladei evidențiază oralitatea textului. În portretizarea celor două personaje simbolice sunt utilizate epitetul și antiteza: Crypto este \textit{„sterp și nărăvaș”}, \textit{„rigă spân”}; lapona e \textit{„mică, liniștită”} și \textit{„prea-cuminte”}. Ambiguitatea este produsă de metaforele insolite: \textit{„Că sufletul nu e fântână/}[...]\textit{/Pahar e gândul, cu otravă”}.

\marginnote{prozodie}[0.3cm]
Alcătuirea prozodică pare destul de riguroasă inițial: catrene cu rimă încrucișată și măsură predominantă de 8 -- 9 silabe.


\subsection{Concluzie}

Poemul \operatitle\ impune o viziune modernă. Interpretarea dată de însuși \operaauthor\ poemului, \textit{„un \textbf{Luceafăr} întors”}, relevă asemănarea cu problematica capodoperei lui Mihai Eminescu, dar poemul modern este totuși \textit{„un Luceafăr cu rolurile inversate și într-un decor de o nebănuită noutate”}, cum remarcă Nicolae Manolescu.


\chapter{Eseu despre particularitățile de construcție a personajului principal dintr-un basm cult studiat}
% Commands
\renewcommand{\operatitle}{\textbfit{„Povestea lui Harap-Alb”}} % title of the text
\renewcommand{\operaauthor}{Ion Creangă} % author of the text


% Beginning of text
\subsection{Context}

\operatitle\ de \operaauthor\ este un basm cult, publicat în revista \textit{„Convorbiri literare”}, în anul 1877.


\section{Precizarea statutului social, psihologic, moral etc. al personajului ales}

\marginnote{erou atipic}[0.8cm]
Harap-Alb este protagonistul basmului, întruchipare a binelui, dar este un erou atipic de basm, deoarece este lipsit de însușiri supranaturale, fiind construit realist, ca o ființă complexă, care învață din greșeli și progresează. De aceea este personaj \textit{„rotund”}, ieșind din stereotipia superiorității mezinului. Este personaj \textit{„tridimensional”}, căci iese din tipar, surprinde, ca, de exemplu, atunci când îi dă calului cu frâul în cap sau râde împreună cu ceilalți de Gerilă, în casa de aramă.

\marginnote{de la naivitate la înțelepciune}[0.8cm]
Statutul inițial al eroului este cel de neinițiat. Mezinul craiului este naiv, nu știe să distingă adevărul de minciună, să vadă caracterul unui om dincolo de aparențe. Are nevoie de experiența vieții spre a dobândi înțelepciune. Se deosebește de frații săi, încă de la început, prin bunătate, calitate răsplătită de sfaturile Sfintei Duminici, după ce o miluiește cu un ban. Deși are calitățile necesare unui viitor împărat, în viziunea autorului, fiind \textit{„cel mai \textbf{vrednic} dintre nepoți”}, cum spune Împăratul Verde, acestea nu sunt individualizate de la început, ci și le descoperă prin intermediul probelor la care este supus, când dovedește generozitate, prietenie, respectare a jurământului, curaj, responsabilitate.

Numele Harap-Alb semnifică sclav-alb, rob de origine nobilă, dar și condiția de învățăcel, faptul de a fi supus inițierii, transformării. Cele trei nume ale lui corespund, în plan compozițional, celor trei etape ale drumului inițiatic: la început -- \textit{„fiul craiului”, mezinul (naivul)}; pe parcursul călătoriei -- \textit{Harap-Alb (învățăcelul)}; la sfârșit -- \textit{împăratul (inițiatul)}.


\section{Ilustrarea elementelor de structură și de compoziție ale basmului, semnificative pentru realizarea personajului din basmul cult studiat {\footnotesize\normalfont (de exemplu: acțiune, conflict, relații temporale și spațiale, incipit, final, tehnici narative, perspectivă narativă, registre stilistice, limbajul personajelor etc.)}}

Titlul sugerează tema basmului: maturizarea mezinului craiului. Concret, eroul parcurge o aventură imaginară, un drum al maturizării, în care dobândește valori morale și etice, pentru ca la final să devină împărat (basmul are, așadar, valoare de Bildungsroman).

Acțiunea se desfășoară linear, prin înlănțuirea secvențelor narative, respectă modelul structural al basmului și implică prezența fabulosului, dar mai puțin în ceea ce-l privește strict pe mezin. Conflictul dintre bine și rău se încheie prin victoria forțelor binelui. Este amplificat procedeul compozițional al triplicării în cazul probelor pe care eroul le are de trecut. Sunt prezente cifre și obiecte magice.

Personajele (oameni, dar și \textit{„ființe himerice”} cu comportament omenesc) îndeplinesc, prin raportare la erou, o serie de funcții (antagonist, ajutoare, donatori), ca în basmul popular, dar sunt individualizate, mai ales, prin limbaj.


\section{Ilustrarea trăsăturilor personajului ales, prin secvențe \\ narative/situații semnificative sau prin citate comentate}

Protagonistul este construit prin procedee de caracterizare directă (de către narator, de către alte personaje și prin autocaracterizare) și de caracterizare indirectă, prin fapte, limbaj, gânduri, relații cu alte personaje, nume.

\marginnote{ajutoare și donatori}[0.8cm]
Eroul este sprijinit de ajutoare și donatori: ființe cu însușiri supranaturale (Sfânta Duminică), animale fabuloase (calul năzdrăvan, crăiasa furnicilor și a albinelor), făpturi himerice (cei cinci tovarăși) sau obiecte miraculoase (aripile crăieselor, smicelele de măr, apa vie, apa moartă) și se confruntă cu răufăcătorul/personajul antagonist (Spânul), care are și funcție de trimițător. Personajul căutat este fata de împărat.

Cu excepția eroului care este văzut în evoluție, de la naivitate la înțelepciune, celelalte personaje au o trăsătură dominantă.

Primele întâlniri cu inițiatorii săi, Sfânta Duminică, apoi calul năzdrăvan și Spânul, pun în lumină naivitatea, incapacitatea de a distinge adevărul de aparențe.

După ce iese din împărăția tatălui său, crăișorul se rătăcește în pădurea-labirint. Încalcă sfatul dat de tată (interdicția să se ferească de omul spân și de omul roș) și își ia drept călăuză un spân viclean. În episodul coborârii în fântână, naratorul surprinde lipsa de experiență a tânărului, prin caracterizare directă. Naivitatea tânărului face posibilă supunerea prin vicleșug.

\marginnote{întâlnirea cu Spânul}[0.8cm]
Antagonistul (răufăcătorul) îl închide pe tânăr în fântână și îi cere, pentru a-l lăsa în viață, să facă schimb de identitate, să devină robul lui și să jure \textit{„pe ascuțișul paloșului”} (sugestie a unui cod al conduitei cavalerești) să-i dea ascultare întru toate, \textit{„până când va muri și iar va învia”}, condiționare paradoxală, dar care arată și calea de eliberare. De asemenea, Spânul îi dă fiului de crai numele de Harap-Alb.

Spânul personifică răul, dar este și inițiatorul pretențios: cu cât încercările la care îl supune pe tânăr sunt mai grele, cu atât eroul dovedește calități morale care conturează portretul viitorului împărat.

Spânul îi cere să aducă \textit{„sălăți”} din Grădina Ursului, pielea cu pietrele prețioase din Pădurea Cerbului și pe fata Împăratului Roș. Harap-Alb își demonstrează curajul și destoinicia în trecerea primelor două probe cu ajutorul obiectelor magice de la Sfânta Duminică.

\marginnote{prietenia}[0.8cm]
Pentru aducerea fetei Împăratului Roș este sprijinit de adjuvanți și donatori. Ca și în cazul milosteniei față de bătrâna cerșetoare, aceste personaje îl ajută pentru că mai întâi el și-a dovedit generozitatea și îndemânarea (față de roiul de albine), bunătatea și curajul (la întâlnirea cu nunta de furnici), prietenia/spiritul de tovărășie (față de Gerilă, Flămânzilă, Setilă, Ochilă și Păsări-Lăți-Lungilă).

Ultima probă presupune mai multe serii de probe, prin care Împăratul Roș tinde să îndepărteze ceata de pețitori (casa încălzită, ospățul, alegerea macului de nisip) și care o vizează direct pe fată (fuga nocturnă a fetei transformată în pasăre, ghicitul fetei/motivul dublului și proba impusă chiar de fată: aducerea unor obiecte magice, \textit{„trei smicele de măr dulce și apă vie și apă moartă de unde se bat munții în capete”}).

\marginnote{onestitatea}[0.3cm]
Pentru erou, aducerea fetei Împăratului Roș la Spân este cea mai dificilă încercare, pentru că pe drum se îndrăgostește de ea, dar, onest, își respectă jurământul făcut și nu-i mărturisește adevărata sa identitate.

La întoarcerea la curtea lui Verde-Împărat are loc recunoașterea și transfigurarea eroului, dar și demascarea și pedepsirea răufăcătorului.

Spânul este demascat de fată, o \textit{„farmazoană”} (are puteri supranaturale). El îi taie capul lui Harap-Alb și îl dezleagă astfel pe erou de jurământul supunerii, semn că inițierea este încheiată, iar calul îl omoară pe răufăcător. Eroul este înviat de fată cu ajutorul obiectelor magice. Învierea este o trecere la o altă identitate: aceea de împărat iubit, slăvit și puternic. Pentru vrednicia lui, primește răsplata cuvenită: nunta și împărăția.


\subsection{Concluzie}

În concluzie, deși este un personaj de basm, eroul nu reprezintă doar tipul voinicului, ca Făt-Frumos din basmele populare, ci este și un \textit{„om de soi bun”}, eroul \textit{„vrednic”} (cum spune Verde-Împărat) care traversează o serie de probe, se maturizează și devine împărat. Basmul poate fi, astfel, considerat un Bildungsroman.



% moara cu noroc
\chapter{Eseu cu privire la tema și viziunea despre lume dintr-o nuvelă studiată}
% Commands
\renewcommand{\operatitle}{\textbfit{„Moara cu noroc”}} % title of the text
\renewcommand{\operaauthor}{Ioan Slavici} % author of the text


% Beginning of text
\subsection{Context}

Publicată în 1881, în volumul de debut \textbfit{„Novele din popor”}, nuvela realistă, de factură psihologică \operatitle\ devine una dintre scrierile reprezentative pentru viziunea lui Ioan Slavici asupra lumii și asupra vieții satului transilvănean.


\section{Evidențierea trăsăturilor care fac posibilă încadrarea nuvelei într-o tipologie, într-un curent cultural/literar, într-o orientare tematică}

\operatitle\ de \operaauthor\ este o nuvelă, adică o specie epică în proză, cu un fir narativ central și o construcție epică riguroasă, cu un conflict concentrat. Personajele relativ puține scot în evidență evoluția personajului principal, puternic individualizat.

\marginnote{nuvelă psihologică}[0.3cm]
Este o nuvelă psihologică deoarece înfățișează frământările de conștiință ale personajului principal, care trăiește un conflict interior, moral și se transformă sufletește, iar analiza se realizează prin tehnici de investigare psihologică: monolog interior, stil indirect liber, scene dialogate, însoțite de notația gesticii și a mimicii.

\marginnote{nuvelă realistă}[0.8cm]
Este o nuvelă realistă prin: tema familiei și a dorinței de înavuțire, obiectivitatea perspectivei narative, includerea de personaje tipice pentru o categorie socială (Ghiță reprezintă tipul cârciumarului dornic de îmbogățire, Pintea este jandarmul, Lică este Sămădăul, dar și tâlharul), verosimilitatea, prezentarea veridică a societății ardelenești din a doua jumătate a secolului al \rom{19}-lea, tehnica detaliului semnificativ în descriere și în portretizare.


\section{Ilustrarea temei nuvelei studiate prin episoade/citate/sec\-ven\-țe comentate}

\operatitle\ de \operaauthor\ are ca temă consecințele nefaste și dezumanizante ale dorinței de îmbogățire. Din perspectiva socială, nuvela prezintă încercarea lui Ghiță de a-și schimba statutul social (din cizmar vrea să devină cârciumar) și de a asigura familiei sale un trai îndestulat. Din perspectivă moralizatoare, nuvela prezintă consecințele nefaste ale dorinței de a avea bani. Din perspectivă psihologică, nuvela prezintă conflictul interior trăit de Ghiță, care, dornic de prosperitate economică, își pierde treptat încrederea în sine și familie.


\section{Prezentarea elementelor de structură și de compoziție ale textului narativ, semnificative pentru tema și viziunea despre lume din nuvela studiată {\footnotesize\normalfont (de exemplu: acțiune, conflict, relații temporale și spațiale, incipit, final, tehnici narative, perspectivă narativă, registre stilistice, limbajul personajelor etc.)}}

Titlul nuvelei este mai degrabă ironic. Toposul ales, cârciuma numită Moara cu noroc, ajunge să însemne, mai curând, Moara cu ghinion, Moara care aduce nenorocirea, deoarece câștigurile obținute aici ascund nelegiuiri.

\marginnote{perspectivă narativă}[-0.2cm]
Perspectiva narativă este obiectivă. Întâmplările din nuvelă sunt relatate la persoana a \rom{3}-a, de către un narator atașat, omniscient și omniprezent.

Pe lângă perspectiva obiectivă a naratorului, apare tehnica punctului de vedere în intervențiile simetrice a bătrânei, din incipitul și finalul nuvelei. Soacra afirmă la început, într-o discuție cu Ghiță, că: \textit{„Omul să fie mulțumit cu sărăcia sa, căci, dacă-i vorba, nu bogăția, ci liniștea colibei tale te face fericit”}, iar la sfârșit pune întâmplările tragice din nuvelă pe seama destinului necruțător: \textit{„așa le-a fost data!..”}.

Cele două teze morale, formulate în prolog și epilog, sunt confirmate în desfășurarea narativă, iar personajele care încalcă aceste principii ale satului tradițional sunt sancționate în finalul nuvelei.

În dialogul din incipitul nuvelei, dintre soacră și Ghiță, se confruntă două concepții despre viață/fericire: bătrâna este adepta valorilor tradiționale, în timp ce Ghiță, capul familiei, dorește bunăstarea materială. Ghiță, cizmar sărac, dar cinstit și harnic, ia în arendă cârciuma de la Moara cu noroc, pentru a câștiga rapid bani, ca să-și deschidă un atelier. Cârciumarul nu este la început un om slab, ci dimpotrivă, voluntar, care își asumă responsabilitatea destinului celorlalți.

\marginnote{conflictul central}[0.3cm]
Fiind o nuvelă psihologică, în \operatitle\ de \operaauthor\ conflictul central este unul moral, psihologic, conflictul interior al protagonistului. Personajul principal, Ghiță, oscilează între dorința de a rămâne om cinstit, pe de o parte, și dorința de a se îmbogăți alături de Lică, pe de altă parte. Conflictul interior se reflectă în plan exterior, în confruntarea dintre cârciumarul Ghiță și Lică Sămădăul.

În nuvela realistă, spațiul și timpul sunt precizate. Cârciuma de la Moara cu noroc este așezată la răscruce de drumuri, izolată, înconjurată de pustietăți întunecoase. Acțiunea se desfășoară pe parcursul unui an, între două repere temporale cu valoare religioasă: de la Sfântul Gheorghe până la Paștele din anul următor; apa și focul purifică locul.

Alcătuită din 17 capitole, cu prolog și epilog, nuvela are un subiect concentrat.

\marginnote{expozițiunea}[0.3cm]
În expozițiune, descrierea drumului care merge la Moara cu noroc și a locului în care se află cârciuma fixează cadrul acțiunii. Ghiță, cizmar sărac, hotărăște să ia în arendă cârciuma de la Moara cu noroc, pentru a câștiga bani mai mulți și mai repede. O vreme, afacerile îi merg bine, iar primele semne ale bunăstării și ale armoniei în care trăiește familia nu întârzie să apară, scena numărării banilor, sâmbătă seara, fiind sugestivă.

\marginnote{intriga}[0.9cm]
Apariția lui Lică Sămădăul, șeful porcarilor și al turmelor de porci din împrejurimi, la Moara cu noroc, constituie intriga nuvelei, declanșând în sufletul lui Ghiță conflictul interior și tulburând echilibrul familiei. Lică îi cere să-i spună cine trece pe la cârciumă, iar Ghiță își dă seama că nu poate rămâne la Moara cu noroc fără acordul Sămădăului.

Mai întâi, Ghiță își ia toate măsurile de apărare împotriva lui Lică: merge la Arad să-și cumpere două pistoale, își face rost de doi câini și își angajează încă o slugă, pe Marți, \textit{„un ungur înalt ca un brad”}.

\marginnote{desfășurarea acțiunii}[1.2cm]
Desfășurarea acțiunii ilustrează procesul înstrăinării cârciumarului față de familie și al dezumanizării provocate de dorința de îmbogățire prin complicitatea cu Lică. Datorită generozității Sămădăului, starea materială a lui Ghiță devine tot mai înfloritoare, numai că Ghiță începe să-și piardă încrederea în sine. Cârciumarul devine interiorizat, mohorât, violent, îi plac jocurile crude, primejdioase, se poartă brutal față de Ana, pe care o protejase până atunci, și față de copii. La un moment dat, ajunge să regrete că are familie și copii, pentru că nu-și poate asuma total riscul îmbogățirii alături de Lică. Cârciumarul se aliază cu jandarmul Pintea, fost hoț de codru și tovarăș al lui Lică, pentru a-l da în vileag pe Sămădău, însă nu este cinstit nici față de acesta, căci dorește să își păstreze o parte din banii obținuți din afaceri necurate.

\marginnote{punctul culminant}[0.8cm]
Punctul culminant ilustrează dezumanizarea lui Ghiță. La sărbătorile Paștelui, Ghiță își aruncă soția în brațele Sămădăului, lăsând-o singură la cârciumă, în timp ce el merge să-l anunțe pe jandarm că Lică are asupra lui bani furați. Dezgustată de lașitatea soțului și neștiind motivul real pentru care acesta plecase, într-un gest de răzbunare disperată, Ana i se dăruiește lui Lică deoarece, spune ea, în ciuda nelegiuirilor comise, el e \textit{„om”}, pe când Ghiță \textit{„nu e decât muiere îmbrăcată în haine bărbătești”}.

\marginnote{deznodământul}[0.8cm]
Deznodământul este tragic. Dându-și seama că soția l-a înșelat, Ghiță o ucide pe Ana, fiind la rândul lui omorât de Răuț, din ordinul lui Lică. Un incendiu provocat de oamenii lui Lică mistuie cârciuma de la Moara cu noroc. Pentru a nu cădea viu în mâinile lui Pintea, Lică se sinucide izbindu-se cu capul de un copac. Nuvela are final moralizator, sancționarea protagoniștilor este pe măsura faptelor. Singurele personaje care supraviețuiesc sunt bătrâna și copiii, ființele morale și inocente.

În nuvelă, accentul nu cade pe actul povestirii, ci pe complexitatea personajelor, care par să aibă un destin prestabilit.

Personajul principal, Ghiță, este cel mai complex din nuvelistica lui Slavici, un personaj \textit{„rotund”}, care trăiește un proces al dezumanizării, cu frământări sufletești și ezitări. Destinul lui ilustrează consecințele nefaste ale dorinței de îmbogățire.

Lică rămâne pe parcursul nuvelei egal cu sine, \textit{„un om rău și primejdios”}. Sămădău și tâlhar, este necruțător cu trădătorii, generos cu aceia care îl sprijină în afacerile necurate, hotărât și crud.

Ana suferă esențiale transformări interioare care îi oferă scriitorului posibilitatea unei fine analize a psihologiei feminine. La început o femeie devotată căminului, protejată mai întâi de mamă și apoi de soț, reprezentând un ideal de feminitate, Ana este împinsă în brațele Sămădăului și apoi este ucisă de Ghiță, fiindcă l-a înșelat.

Trăsăturile personajelor se desprind din fapte, vorbe, gesturi și din relațiile care se stabilesc între acestea. De asemenea, naratorul realizează portrete sugestive, iar detaliile fizice relevă trăsături morale sau statutul social. Mijloacele de investigație psihologică sunt: scenele dialogate, monologul interior de factură tradițională și acela realizat în stil indirect liber, introspecția, notația gesticii, a mimicii și a tonului vocii.


\subsection{Concluzie}

\operatitle\ de \operaauthor\ este o nuvelă realistă și o nuvelă psihologică, pentru că urmărește conflictul interior, frământările în planul conștiinței personajelor.


\chapter{Eseu despre particularitățile de construcție a personajului principal dintr-o nuvelă studiată}
% Commands
\renewcommand{\operatitle}{\textbfit{„Moara cu noroc”}} % title of the text
\renewcommand{\operaauthor}{Ioan Slavici} % author of the text


% Beginning of text
\subsection{Context}

Publicată în 1881, în volumul de debut \textbfit{„Novele din popor”}, nuvela realistă, de factură psihologică \operatitle\ devine una dintre scrierile reprezentative pentru viziunea lui Ioan Slavici asupra lumii și asupra vieții satului transilvănean.


\section{Ilustrarea elementelor de structură și de compoziție ale nuvelei, semnificative pentru realizarea personajului din nuvela studiată {\footnotesize\normalfont (de exemplu: acțiune, conflict, relații temporale și spațiale, incipit, final, tehnici narative, perspectivă narativă, registre stilistice, limbajul personajelor etc.)}}

\operatitle\ de \operaauthor\ are ca temă consecințele nefaste și dezumanizante ale dorinței de îmbogățire.

Perspectiva narativă este obiectivă. Întâmplările din nuvelă sunt relatate la persoana a \rom{3}-a, din perspectiva unui narator omniscient și omniprezent.

Titlul nuvelei este mai degrabă ironic. Toposul ales, cârciuma numită Moara cu noroc, ajunge să însemne, de fapt, Moara cu ghinion, Moara care aduce nenorocirea, deoarece câștigurile obținute aici ascund nelegiuiri și crime.

\marginnote{incipit/\ final}[0.3cm]
Simetria dintre incipitul și finalul nuvelei este dată de vorbele bătrânei, soacra lui Ghiță, care la început susține că: \textit{„Omul să fie mulțumit cu sărăcia sa, căci, dacă-i vorba, nu bogăția, ci liniștea colibei tale te face fericit”}, iar la sfârșit pune întâmplările tragice pe seama destinului necruțător: \textit{”așa le-a fost data!...”}. Cele două afirmații sunt principalele teze morale ale nuvelei.

\marginnote{conflictul central}[0.3cm]
Fiind o nuvelă psihologică, în \operatitle\ de \operaauthor\ conflictul central este moral, psihologic, conflictul interior al protagonistului. Personajul principal, Ghiță, oscilează între dorința de a rămâne om cinstit, pe de o parte, și dorința de a se îmbogăți alături de Lică, pe de altă parte. Conflictul interior se reflectă în plan exterior, în confruntarea dintre cârciumarul Ghiță și Lică Sămădăul.

Stilul nuvelei este sobru, concis, fără podoabe. Limbajul naratorului și al personajelor valorifică aceleași registre stilistice: limbajul regional, ardelenesc, limbajul popular, oralitatea.


\section{Precizarea statutului social, psihologic, moral etc. al personajului ales}

Ghiță este cel mai complex personaj din nuvelistica lui Slavici, un personaj \textit{„rotund”}, al cărui destin ilustrează consecințele nefaste ale dorinței de îmbogățire.

\marginnote{statut inițial}[0.5cm]
Statutul inițial al personajului este reliefat în dialogul din incipitul nuvelei, dintre soacră și Ghiță, în care se confruntă două concepții despre viață: bătrâna este adepta valorilor tradiționale, în timp ce Ghiță, capul familiei, dorește bunăstarea materială. Ghiță, cizmar sărac, dar om harnic, blând și cumsecade, soț iubitor, ia în arendă cârciuma de la Moara cu noroc, pentru a câștiga rapid bani, ca să-și deschidă un atelier.

Atâta timp cât este un om de acțiune, cu inițiativă, lucrurile merg bine. Cârciuma aduce profit, familia trăiește în armonie.


\section{Ilustrarea trăsăturilor personajului ales, prin secvențe narative/situații semnificative sau prin citate comentate}

\marginnote{portretul personajului}[0.3cm]
Trăsăturile personajelor se desprind din fapte, vorbe, gesturi și din relațiile care se stabilesc între acestea (caracterizare indirectă). De asemenea, naratorul realizează portrete sugestive (caracterizare directă). Portretul fizic al lui Ghiță este aproape absent: este redus la câteva detalii, la început (\textit{„înalt și spătos”}), pentru ca, mai apoi, trăsăturile cârciumarului (expresia chipului, ton, voce etc.) să reflecte transformările sale sufletești.

\marginnote{mijloace de investigație psihologică}[-0.2cm]
Pentru portretul moral al personajului principal, Slavici a folosit mijloace de investigație psihologică, precum: scenele dialogate, monologul interior de factură tradițională și acela realizat în stil indirect liber, introspecția, notația gesticii, a mimicii și a tonului vocii.

Procesul de înstrăinare a lui Ghiță față de familie începe din momentul venirii lui Lică la cârciumă. La început, Ghiță își ia toate măsurile de apărare împotriva lui Lică: merge la Arad să-și cumpere două pistoale, își face rost de doi câini și își angajează încă o slugă, pe Marți. Deși înțelege că Lică reprezintă un pericol, Ghiță nu se poate sustrage ispitei câștigului, mai ales că își dă seama că nu poate rămâne la Moara cu noroc fără acordul Sămădăului.

\marginnote{patima pentru bani}[-0.2cm]
Bun cunoscător de oameni, Lică se folosește de patima lui Ghiță pentru bani spre a-l atrage pe acesta în afacerile lui necurate și apoi pentru a-i anula personalitatea.

Gesturile și gândurile cârciumarului trădează frământările sale sufletești, conflictul său interior, și contribuie la realizarea analizei psihologice.

\marginnote{lașitate}[1.3cm]
La un moment dat, Ghiță ajunge să regrete faptul că are familie și că nu-și poate asuma total riscul îmbogățirii alături de Lică. Prin intermediul monologului interior sunt redate gândurile și frământările personajului, realizându-se în felul acesta autocaracterizarea: \textit{„Ei! Ce să-mi fac!? Așa m-a lăsat Dumnezeu! Ce să-mi fac dacă e în mine ceva mai tare decât voința mea?”}.

Ghiță este caracterizat în mod direct de Lică. Acesta își dă seama că Ghiță e om de nădejde și chiar îi spune acest lucru: \textit{„Tu ești om, Ghiță, om cu multă ură în sufletul tău, și ești om cu minte”}.

\marginnote{înstrăinare de familie}[-0.1cm]
Axa vieții morale a personajului este distrusă treptat; se simte înstrăinat de toți și de toate. De rușinea lumii, de dragul soției și al copiilor, se gândește că ar fi mai bine să plece de la Moara cu noroc. Începe să colaboreze cu Pintea, dar nu este sincer în totalitate nici față de acesta.

\marginnote{degradare morală}[0.8cm]
Ghiță ajunge pe ultima treaptă a degradării morale în momentul în care, orbit de furie și dispus să facă orice pentru a se răzbuna pe Lică, își aruncă soția, drept momeală, în brațele Sămădăului. Speră până în ultimul moment că Ana va rezista influenței malefice a lui Lică. Dezgustată însă de lașitatea lui Ghiță care se înstrăinase de ea și de familie, într-un gest de răzbunare, Ana i se dăruiește lui Lică, deoarece, spune ea, în ciuda nelegiuirilor comise, Lică e \textit{„om”}, pe când Ghiță \textit{„nu e decât muiere îmbrăcată în haine bărbătești”} (caracterizare directă).

Dându-și seama că soția l-a înșelat, Ghiță o ucide pe Ana. Din ordinul lui Lică, Ghiță este omorât de Răuț, iar cârciuma incendiată.


\subsection{Concluzie}

\operatitle\ de \operaauthor\ este o nuvelă realistă, psihologică, pentru că urmărește efectele dorinței de îmbogățire, frământările personajelor în planul conștiinței, conflictul interior.



% baltagul
\chapter{Eseu privind tema și viziunea despre lume dintr-un roman interbelic, realist -- mitic}
% Commands
\renewcommand{\operatitle}{\textbfit{„Riga Crypto și lapona Enigel”}} % title of the text
\renewcommand{\operaauthor}{Ion Barbu} % author of the text


% Beginning of text
\subsection{Context}

Publicată în 1924, integrată apoi în volumul \textbfit{„Joc secund”}, balada \operatitle\ face parte din a doua etapă de creație barbiană, numită baladic-orientală, dar anunță dezvoltarea ulterioară a poeziei lui \operaauthor.


\section{Evidențierea trăsăturilor care fac posibilă încadrarea poeziei studiate într-o tipologie, într-un curent cultural/literar, într-o orientare tematică}

\marginnote{poem alegoric}[1.2cm]
\operatitle\ este subintitulată \textit{„Baladă”}, începe ca un cântec bătrânesc de nuntă, dar se realizează în viziune modernă, ca un amplu poem de cunoaștere și poem alegoric, o poveste de iubire din lumea vegetală. Autorul păstrează din specia tradițională schema epică și personajele antagonice, dar evenimentele narate sunt de natură fantastică (dialogul în vis dintre rigă și laponă) și alegorică. Scenariul epic este dublat de caracterul dramatic și de \textit{„lirismul de măști”}, personajele având semnificații simbolice multiple (materia și spiritul etc.).

Poemul se încadrează modernismului interbelic prin intelectualizarea emoției, imaginar poetic inedit, ambiguitate, metafore surprinzătoare și cuvinte cu sonorități neobișnuite, înnoiri prozodice.


\section{Prezentarea imaginilor/ideilor poetice, relevante pentru te\-ma și viziunea despre lume din textul studiat}

Tema poeziei o reprezintă iubirea ca modalitate de cunoaștere a lumii. Fiind „un \textbfit{Luceafăr} întors”, poemul prezintă drama cunoașterii și a incompatibilității dintre două lumi (regnuri).

Titlul baladei trimite cu gândul la marile povești de dragoste din literatura universală, \textbfit{„Romeo și Julieta”}, \textbfit{„Tristan și Isolda”}. Însă la \operaauthor, membrii cuplului sunt antagonici (fac parte din regnuri diferite). Sunt personaje romantice cu trăsături excepționale, dar negative în raport cu norma comună (Crypto e \textit{„sterp”} și \textit{„nărăvaș/Că nu voia să înflorească”}, iar Enigel este \textit{„prea-cuminte”}).

\marginnote{semnificația titlului}[-0.3cm]
Numele Crypto are dublă semnificație: cel tăinuit, \textit{„inimă ascunsă”}, provenind din adjectivul \textit{„criptic”}, (\textit{„ascuns”}, \textit{„tăinuit”}), dar sugerează, în egală măsură, apartenența sa la familia ciupercilor, numele științific \textit{„criptogame”}. Personajul este rege (rigă) al făpturilor inferioare, din regnul vegetal. Numele cu sonoritate nordică Enigel sugerează originea laponei (de la pol) și trimite probabil la semnificația cuvântului din limba suedeză, \textit{„înger”} (care provine din latinescul \textit{„angelus”}).


\section{Ilustrarea elementelor de compoziție și de limbaj ale textului poetic studiat, semnificative pentru tema și viziunea despre lume {\footnotesize\normalfont(de exemplu: imaginar poetic, titlu, incipit, relații de opoziție și de simetrie, motiv poetic, laitmotiv, figuri semantice/tropi, elemente de prozodie etc.)}}

\marginnote{compoziție}[0.3cm]
La nivel formal, poezia este alcătuită din două părți, fiecare dintre ele prezentând câte o nuntă: una împlinită, cadru al celeilalte nunți, povestită, ratată, modificată în final prin căsătoria lui Crypto cu măsălarița. Formula compozițională este aceea a povestirii în ramă.

Prologul conturează în puține imagini atmosfera de la finalul unei nunți trăite. Primele patru strofe constituie rama viitoarei povești și reprezintă dialogul menestrelului cu \textit{„nuntașul fruntaș”}.

Partea a doua prezintă povestea de iubire neîmplinită dintre Enigel și riga Crypto. Nunta povestită cuprinde mai multe tablouri poetice: portretul și împărăția rigăi Crypto (strofele 5 -- 7), portretul, locurile natale și oprirea din drum a laponei Enigel (strofele 8 -- 9), întâlnirea dintre cei doi (strofa 10), cele trei chemări ale rigăi și primele două refuzuri ale laponei (strofele 11 -- 15), răspunsul laponei și refuzul categoric cu relevarea relației dintre simbolul solar și propria condiție (strofele 16 -- 20), încheierea întâlnirii (strofele 21 -- 22), pedepsirea rigăi în finalul baladei (strofele 23 -- 27). Modurile de expunere sunt, în ordine: descrierea, dialogul și narațiunea.

În expozițiune, sunt prezentate în antiteză portretele membrilor cuplului și locurile lor natale, deosebirile dintre ei generând intriga.

Riga Crypto, \textit{„inimă ascunsă”}, este craiul bureților, căruia dragostea pentru Enigel, \textit{„laponă mică, liniștită”}, îi este fatală. Singura lor asemănare este statutul superior în interiorul propriei lumi: el este rigă al plantelor inferioare, care nu înfloresc, iar păstorița care își conduce turmele de reni spre sud este o stăpână a regnului animal, în ipostaza de ființă rațională, omul -- \textit{„fiară bătrână”}.

\marginnote{imaginar poetic}[0.3cm]
Spațiul definitoriu al existenței, pentru Crypto, este umezeala perpetuă și impură, în timp ce lapona vine \textit{„din țări de gheață urgisită”}, spațiu rece, ceea ce explică aspirația ei spre soare și lumină, dar și mișcarea de transhumanță care ocazionează popasul în ținutul rigăi.

Membrii cuplului fac parte din regnuri diferite și, de aceea, nu pot comunica în plan real. Întâlnirea lor se realizează în visul fetei, la fel ca în \textbfit{„Luceafărul”}. Riga este cel care rostește de trei ori descântecul de dragoste și, de tot atâtea ori, lapona îl respinge. Povestea propriu-zisă se dovedește a fi fantastică, ca și în poemul eminescian, doar că rolurile sunt inversate. În dialogul lor, formulele de adresare sugerează familiaritate, afecțiune blândă: repetiția \textit{„Enigel, Enigel”}, epitetul \textit{„rigă blând”}.

În prima chemare-descântec, cu rezonanțe de incantație magică, Crypto își îmbie aleasa cu \textit{„dulceață”} și cu \textit{„fulgi”}, elemente ale existenței sale vegetative, dar care aici capătă conotații erotice. Darul lui este refuzat categoric de Enigel: \textit{„Eu mă duc să culeg/Fragii fragezi mai la vale”}. Refuzul laponei îl pune într-o situație dilematică, dar opțiunea lui e fermă și merge până la sacrificiul de sine, în a doua chemare: \textit{„Dacă pleci să culegi/Începi, rogu-te, cu mine”}.

Al doilea refuz este susținut de enumerarea atributelor lui Crypto: \textit{„blând”}, \textit{„plăpând”}, \textit{necopt -- „Lasă. Așteaptă de te coace”}. Opoziția \textit{„copt”} -- \textit{„necopt”}, reluată în al treilea refuz prin antiteza \textit{soare-umbră}, pune în evidență incompatibilitatea lor. Imaginii de fragilitate a lui Crypto lapona îi opune aspirația ei spre absolut. Soarele este simbolul existenței spirituale, al împlinirii umane, în antiteză cu \textit{„umbra”}, simbol al existenței instinctuale, sterile, vegetative.

Pentru a-și continua drumul către soare și cunoaștere, lapona refuză descântecul rigăi, deși regretă și plânge. Descântecul se întoarce în mod brutal asupra celui care l-a rostit și-l distruge. Făptura firavă este distrusă de propriul vis, cade victimă neputinței și îndrăznelii de a-și depăși limitele.

Finalul este trist. Riga Crypto se transformă într-o ciupercă otrăvitoare, obligat să rămână alături de făpturi asemenea lui, \textit{„Laurul-Balaurul”} și \textit{„măsălarița-mireasă”}. Încercarea ființei inferioare de a-și depăși limitele este pedepsită cu nebunia.

\marginnote{figuri semantice/ tropi}[1.3cm]
Soarele, simbolul spiritului, este imaginat în poem prin metaforele \textit{„roata albă”} (perfecțiunea geometrică) și \textit{„aprins inel”} (simbolul nunții), în antiteză cu \textit{„umbra”}, iar metafora \textit{„sufletul-fântână”} sugerează puritatea, setea de cunoaștere, veșnicia, fiind în antiteză cu \textit{„carnea”} (trupul, instinctele). Spiritul și sufletul sunt atribute ale ființei raționale, înțelepte. Făpturile inferioare care aspiră să dobândească spiritualitate sunt distruse de propriul vis, așa cum i se întâmplă lui Crypto, care „înnebunește” și se transformă în ciupercă otrăvitoare.

Trei mituri fundamentale de origine greacă sunt valorificate în opera poetului: al soarelui (absolutul), al nunții și al oglinzii.

Drumul spre sud al laponei are semnificația unui drum inițiatic, iar popasul în ținutul rigăi este o probă, trecută prin respingerea nunții pe o treaptă inferioară.
\marginnote{semnificații}[0.3cm]
Drumul trece prin cercul Venerii (iubirea ce reduce omul la ipostază de ființă instinctuală), apoi sufletul trebuie să mai urce o treaptă, cercul lui Mercur, mai pur, al intelectului, al cunoașterii raționale. Inițierea completă are loc prin adevărata \textit{nuntă} a trupului și spiritului cu însuși focarul vieții, prin trecerea omului în cercul Soarelui (cunoașterea absolută). Aspirația solară a laponei sugerează faptul că, în momentul întâlnirii cu riga Crypto, aceasta se află pe treapta lui Mercur, fără ca ea să fi trăit experiența iubirii. Chemările lui Crypto o atrag spre cercul Venerii. Ea trăiește iubirea ca experiență inițiatică, dar alege să-și urmeze drumul spre Soare (cunoașterea absolută).

\marginnote{figuri de stil}[0.8cm]
Sub raport stilistic, prezența inversiunilor (\textit{„zice-l-aș”}) și a vocativelor în prima parte a baladei evidențiază oralitatea textului. În portretizarea celor două personaje simbolice sunt utilizate epitetul și antiteza: Crypto este \textit{„sterp și nărăvaș”}, \textit{„rigă spân”}; lapona e \textit{„mică, liniștită”} și \textit{„prea-cuminte”}. Ambiguitatea este produsă de metaforele insolite: \textit{„Că sufletul nu e fântână/}[...]\textit{/Pahar e gândul, cu otravă”}.

\marginnote{prozodie}[0.3cm]
Alcătuirea prozodică pare destul de riguroasă inițial: catrene cu rimă încrucișată și măsură predominantă de 8 -- 9 silabe.


\subsection{Concluzie}

Poemul \operatitle\ impune o viziune modernă. Interpretarea dată de însuși \operaauthor\ poemului, \textit{„un \textbf{Luceafăr} întors”}, relevă asemănarea cu problematica capodoperei lui Mihai Eminescu, dar poemul modern este totuși \textit{„un Luceafăr cu rolurile inversate și într-un decor de o nebănuită noutate”}, cum remarcă Nicolae Manolescu.


\chapter{Particularități de construcție a personajului principal din\-tr-un roman interbelic, realist}
% Commands
\renewcommand{\operatitle}{\textbfit{„Riga Crypto și lapona Enigel”}} % title of the text
\renewcommand{\operaauthor}{Ion Barbu} % author of the text


% Beginning of text
\subsection{Context}

Publicată în 1924, integrată apoi în volumul \textbfit{„Joc secund”}, balada \operatitle\ face parte din a doua etapă de creație barbiană, numită baladic-orientală, dar anunță dezvoltarea ulterioară a poeziei lui \operaauthor.


\section{Evidențierea trăsăturilor care fac posibilă încadrarea poeziei studiate într-o tipologie, într-un curent cultural/literar, într-o orientare tematică}

\marginnote{poem alegoric}[1.2cm]
\operatitle\ este subintitulată \textit{„Baladă”}, începe ca un cântec bătrânesc de nuntă, dar se realizează în viziune modernă, ca un amplu poem de cunoaștere și poem alegoric, o poveste de iubire din lumea vegetală. Autorul păstrează din specia tradițională schema epică și personajele antagonice, dar evenimentele narate sunt de natură fantastică (dialogul în vis dintre rigă și laponă) și alegorică. Scenariul epic este dublat de caracterul dramatic și de \textit{„lirismul de măști”}, personajele având semnificații simbolice multiple (materia și spiritul etc.).

Poemul se încadrează modernismului interbelic prin intelectualizarea emoției, imaginar poetic inedit, ambiguitate, metafore surprinzătoare și cuvinte cu sonorități neobișnuite, înnoiri prozodice.


\section{Prezentarea imaginilor/ideilor poetice, relevante pentru te\-ma și viziunea despre lume din textul studiat}

Tema poeziei o reprezintă iubirea ca modalitate de cunoaștere a lumii. Fiind „un \textbfit{Luceafăr} întors”, poemul prezintă drama cunoașterii și a incompatibilității dintre două lumi (regnuri).

Titlul baladei trimite cu gândul la marile povești de dragoste din literatura universală, \textbfit{„Romeo și Julieta”}, \textbfit{„Tristan și Isolda”}. Însă la \operaauthor, membrii cuplului sunt antagonici (fac parte din regnuri diferite). Sunt personaje romantice cu trăsături excepționale, dar negative în raport cu norma comună (Crypto e \textit{„sterp”} și \textit{„nărăvaș/Că nu voia să înflorească”}, iar Enigel este \textit{„prea-cuminte”}).

\marginnote{semnificația titlului}[-0.3cm]
Numele Crypto are dublă semnificație: cel tăinuit, \textit{„inimă ascunsă”}, provenind din adjectivul \textit{„criptic”}, (\textit{„ascuns”}, \textit{„tăinuit”}), dar sugerează, în egală măsură, apartenența sa la familia ciupercilor, numele științific \textit{„criptogame”}. Personajul este rege (rigă) al făpturilor inferioare, din regnul vegetal. Numele cu sonoritate nordică Enigel sugerează originea laponei (de la pol) și trimite probabil la semnificația cuvântului din limba suedeză, \textit{„înger”} (care provine din latinescul \textit{„angelus”}).


\section{Ilustrarea elementelor de compoziție și de limbaj ale textului poetic studiat, semnificative pentru tema și viziunea despre lume {\footnotesize\normalfont(de exemplu: imaginar poetic, titlu, incipit, relații de opoziție și de simetrie, motiv poetic, laitmotiv, figuri semantice/tropi, elemente de prozodie etc.)}}

\marginnote{compoziție}[0.3cm]
La nivel formal, poezia este alcătuită din două părți, fiecare dintre ele prezentând câte o nuntă: una împlinită, cadru al celeilalte nunți, povestită, ratată, modificată în final prin căsătoria lui Crypto cu măsălarița. Formula compozițională este aceea a povestirii în ramă.

Prologul conturează în puține imagini atmosfera de la finalul unei nunți trăite. Primele patru strofe constituie rama viitoarei povești și reprezintă dialogul menestrelului cu \textit{„nuntașul fruntaș”}.

Partea a doua prezintă povestea de iubire neîmplinită dintre Enigel și riga Crypto. Nunta povestită cuprinde mai multe tablouri poetice: portretul și împărăția rigăi Crypto (strofele 5 -- 7), portretul, locurile natale și oprirea din drum a laponei Enigel (strofele 8 -- 9), întâlnirea dintre cei doi (strofa 10), cele trei chemări ale rigăi și primele două refuzuri ale laponei (strofele 11 -- 15), răspunsul laponei și refuzul categoric cu relevarea relației dintre simbolul solar și propria condiție (strofele 16 -- 20), încheierea întâlnirii (strofele 21 -- 22), pedepsirea rigăi în finalul baladei (strofele 23 -- 27). Modurile de expunere sunt, în ordine: descrierea, dialogul și narațiunea.

În expozițiune, sunt prezentate în antiteză portretele membrilor cuplului și locurile lor natale, deosebirile dintre ei generând intriga.

Riga Crypto, \textit{„inimă ascunsă”}, este craiul bureților, căruia dragostea pentru Enigel, \textit{„laponă mică, liniștită”}, îi este fatală. Singura lor asemănare este statutul superior în interiorul propriei lumi: el este rigă al plantelor inferioare, care nu înfloresc, iar păstorița care își conduce turmele de reni spre sud este o stăpână a regnului animal, în ipostaza de ființă rațională, omul -- \textit{„fiară bătrână”}.

\marginnote{imaginar poetic}[0.3cm]
Spațiul definitoriu al existenței, pentru Crypto, este umezeala perpetuă și impură, în timp ce lapona vine \textit{„din țări de gheață urgisită”}, spațiu rece, ceea ce explică aspirația ei spre soare și lumină, dar și mișcarea de transhumanță care ocazionează popasul în ținutul rigăi.

Membrii cuplului fac parte din regnuri diferite și, de aceea, nu pot comunica în plan real. Întâlnirea lor se realizează în visul fetei, la fel ca în \textbfit{„Luceafărul”}. Riga este cel care rostește de trei ori descântecul de dragoste și, de tot atâtea ori, lapona îl respinge. Povestea propriu-zisă se dovedește a fi fantastică, ca și în poemul eminescian, doar că rolurile sunt inversate. În dialogul lor, formulele de adresare sugerează familiaritate, afecțiune blândă: repetiția \textit{„Enigel, Enigel”}, epitetul \textit{„rigă blând”}.

În prima chemare-descântec, cu rezonanțe de incantație magică, Crypto își îmbie aleasa cu \textit{„dulceață”} și cu \textit{„fulgi”}, elemente ale existenței sale vegetative, dar care aici capătă conotații erotice. Darul lui este refuzat categoric de Enigel: \textit{„Eu mă duc să culeg/Fragii fragezi mai la vale”}. Refuzul laponei îl pune într-o situație dilematică, dar opțiunea lui e fermă și merge până la sacrificiul de sine, în a doua chemare: \textit{„Dacă pleci să culegi/Începi, rogu-te, cu mine”}.

Al doilea refuz este susținut de enumerarea atributelor lui Crypto: \textit{„blând”}, \textit{„plăpând”}, \textit{necopt -- „Lasă. Așteaptă de te coace”}. Opoziția \textit{„copt”} -- \textit{„necopt”}, reluată în al treilea refuz prin antiteza \textit{soare-umbră}, pune în evidență incompatibilitatea lor. Imaginii de fragilitate a lui Crypto lapona îi opune aspirația ei spre absolut. Soarele este simbolul existenței spirituale, al împlinirii umane, în antiteză cu \textit{„umbra”}, simbol al existenței instinctuale, sterile, vegetative.

Pentru a-și continua drumul către soare și cunoaștere, lapona refuză descântecul rigăi, deși regretă și plânge. Descântecul se întoarce în mod brutal asupra celui care l-a rostit și-l distruge. Făptura firavă este distrusă de propriul vis, cade victimă neputinței și îndrăznelii de a-și depăși limitele.

Finalul este trist. Riga Crypto se transformă într-o ciupercă otrăvitoare, obligat să rămână alături de făpturi asemenea lui, \textit{„Laurul-Balaurul”} și \textit{„măsălarița-mireasă”}. Încercarea ființei inferioare de a-și depăși limitele este pedepsită cu nebunia.

\marginnote{figuri semantice/ tropi}[1.3cm]
Soarele, simbolul spiritului, este imaginat în poem prin metaforele \textit{„roata albă”} (perfecțiunea geometrică) și \textit{„aprins inel”} (simbolul nunții), în antiteză cu \textit{„umbra”}, iar metafora \textit{„sufletul-fântână”} sugerează puritatea, setea de cunoaștere, veșnicia, fiind în antiteză cu \textit{„carnea”} (trupul, instinctele). Spiritul și sufletul sunt atribute ale ființei raționale, înțelepte. Făpturile inferioare care aspiră să dobândească spiritualitate sunt distruse de propriul vis, așa cum i se întâmplă lui Crypto, care „înnebunește” și se transformă în ciupercă otrăvitoare.

Trei mituri fundamentale de origine greacă sunt valorificate în opera poetului: al soarelui (absolutul), al nunții și al oglinzii.

Drumul spre sud al laponei are semnificația unui drum inițiatic, iar popasul în ținutul rigăi este o probă, trecută prin respingerea nunții pe o treaptă inferioară.
\marginnote{semnificații}[0.3cm]
Drumul trece prin cercul Venerii (iubirea ce reduce omul la ipostază de ființă instinctuală), apoi sufletul trebuie să mai urce o treaptă, cercul lui Mercur, mai pur, al intelectului, al cunoașterii raționale. Inițierea completă are loc prin adevărata \textit{nuntă} a trupului și spiritului cu însuși focarul vieții, prin trecerea omului în cercul Soarelui (cunoașterea absolută). Aspirația solară a laponei sugerează faptul că, în momentul întâlnirii cu riga Crypto, aceasta se află pe treapta lui Mercur, fără ca ea să fi trăit experiența iubirii. Chemările lui Crypto o atrag spre cercul Venerii. Ea trăiește iubirea ca experiență inițiatică, dar alege să-și urmeze drumul spre Soare (cunoașterea absolută).

\marginnote{figuri de stil}[0.8cm]
Sub raport stilistic, prezența inversiunilor (\textit{„zice-l-aș”}) și a vocativelor în prima parte a baladei evidențiază oralitatea textului. În portretizarea celor două personaje simbolice sunt utilizate epitetul și antiteza: Crypto este \textit{„sterp și nărăvaș”}, \textit{„rigă spân”}; lapona e \textit{„mică, liniștită”} și \textit{„prea-cuminte”}. Ambiguitatea este produsă de metaforele insolite: \textit{„Că sufletul nu e fântână/}[...]\textit{/Pahar e gândul, cu otravă”}.

\marginnote{prozodie}[0.3cm]
Alcătuirea prozodică pare destul de riguroasă inițial: catrene cu rimă încrucișată și măsură predominantă de 8 -- 9 silabe.


\subsection{Concluzie}

Poemul \operatitle\ impune o viziune modernă. Interpretarea dată de însuși \operaauthor\ poemului, \textit{„un \textbf{Luceafăr} întors”}, relevă asemănarea cu problematica capodoperei lui Mihai Eminescu, dar poemul modern este totuși \textit{„un Luceafăr cu rolurile inversate și într-un decor de o nebănuită noutate”}, cum remarcă Nicolae Manolescu.



% ion
\chapter{Eseu cu privire la tema și viziunea despre lume dintr-un roman interbelic realist, obiectiv}
% Commands
\renewcommand{\operatitle}{\textbfit{„Riga Crypto și lapona Enigel”}} % title of the text
\renewcommand{\operaauthor}{Ion Barbu} % author of the text


% Beginning of text
\subsection{Context}

Publicată în 1924, integrată apoi în volumul \textbfit{„Joc secund”}, balada \operatitle\ face parte din a doua etapă de creație barbiană, numită baladic-orientală, dar anunță dezvoltarea ulterioară a poeziei lui \operaauthor.


\section{Evidențierea trăsăturilor care fac posibilă încadrarea poeziei studiate într-o tipologie, într-un curent cultural/literar, într-o orientare tematică}

\marginnote{poem alegoric}[1.2cm]
\operatitle\ este subintitulată \textit{„Baladă”}, începe ca un cântec bătrânesc de nuntă, dar se realizează în viziune modernă, ca un amplu poem de cunoaștere și poem alegoric, o poveste de iubire din lumea vegetală. Autorul păstrează din specia tradițională schema epică și personajele antagonice, dar evenimentele narate sunt de natură fantastică (dialogul în vis dintre rigă și laponă) și alegorică. Scenariul epic este dublat de caracterul dramatic și de \textit{„lirismul de măști”}, personajele având semnificații simbolice multiple (materia și spiritul etc.).

Poemul se încadrează modernismului interbelic prin intelectualizarea emoției, imaginar poetic inedit, ambiguitate, metafore surprinzătoare și cuvinte cu sonorități neobișnuite, înnoiri prozodice.


\section{Prezentarea imaginilor/ideilor poetice, relevante pentru te\-ma și viziunea despre lume din textul studiat}

Tema poeziei o reprezintă iubirea ca modalitate de cunoaștere a lumii. Fiind „un \textbfit{Luceafăr} întors”, poemul prezintă drama cunoașterii și a incompatibilității dintre două lumi (regnuri).

Titlul baladei trimite cu gândul la marile povești de dragoste din literatura universală, \textbfit{„Romeo și Julieta”}, \textbfit{„Tristan și Isolda”}. Însă la \operaauthor, membrii cuplului sunt antagonici (fac parte din regnuri diferite). Sunt personaje romantice cu trăsături excepționale, dar negative în raport cu norma comună (Crypto e \textit{„sterp”} și \textit{„nărăvaș/Că nu voia să înflorească”}, iar Enigel este \textit{„prea-cuminte”}).

\marginnote{semnificația titlului}[-0.3cm]
Numele Crypto are dublă semnificație: cel tăinuit, \textit{„inimă ascunsă”}, provenind din adjectivul \textit{„criptic”}, (\textit{„ascuns”}, \textit{„tăinuit”}), dar sugerează, în egală măsură, apartenența sa la familia ciupercilor, numele științific \textit{„criptogame”}. Personajul este rege (rigă) al făpturilor inferioare, din regnul vegetal. Numele cu sonoritate nordică Enigel sugerează originea laponei (de la pol) și trimite probabil la semnificația cuvântului din limba suedeză, \textit{„înger”} (care provine din latinescul \textit{„angelus”}).


\section{Ilustrarea elementelor de compoziție și de limbaj ale textului poetic studiat, semnificative pentru tema și viziunea despre lume {\footnotesize\normalfont(de exemplu: imaginar poetic, titlu, incipit, relații de opoziție și de simetrie, motiv poetic, laitmotiv, figuri semantice/tropi, elemente de prozodie etc.)}}

\marginnote{compoziție}[0.3cm]
La nivel formal, poezia este alcătuită din două părți, fiecare dintre ele prezentând câte o nuntă: una împlinită, cadru al celeilalte nunți, povestită, ratată, modificată în final prin căsătoria lui Crypto cu măsălarița. Formula compozițională este aceea a povestirii în ramă.

Prologul conturează în puține imagini atmosfera de la finalul unei nunți trăite. Primele patru strofe constituie rama viitoarei povești și reprezintă dialogul menestrelului cu \textit{„nuntașul fruntaș”}.

Partea a doua prezintă povestea de iubire neîmplinită dintre Enigel și riga Crypto. Nunta povestită cuprinde mai multe tablouri poetice: portretul și împărăția rigăi Crypto (strofele 5 -- 7), portretul, locurile natale și oprirea din drum a laponei Enigel (strofele 8 -- 9), întâlnirea dintre cei doi (strofa 10), cele trei chemări ale rigăi și primele două refuzuri ale laponei (strofele 11 -- 15), răspunsul laponei și refuzul categoric cu relevarea relației dintre simbolul solar și propria condiție (strofele 16 -- 20), încheierea întâlnirii (strofele 21 -- 22), pedepsirea rigăi în finalul baladei (strofele 23 -- 27). Modurile de expunere sunt, în ordine: descrierea, dialogul și narațiunea.

În expozițiune, sunt prezentate în antiteză portretele membrilor cuplului și locurile lor natale, deosebirile dintre ei generând intriga.

Riga Crypto, \textit{„inimă ascunsă”}, este craiul bureților, căruia dragostea pentru Enigel, \textit{„laponă mică, liniștită”}, îi este fatală. Singura lor asemănare este statutul superior în interiorul propriei lumi: el este rigă al plantelor inferioare, care nu înfloresc, iar păstorița care își conduce turmele de reni spre sud este o stăpână a regnului animal, în ipostaza de ființă rațională, omul -- \textit{„fiară bătrână”}.

\marginnote{imaginar poetic}[0.3cm]
Spațiul definitoriu al existenței, pentru Crypto, este umezeala perpetuă și impură, în timp ce lapona vine \textit{„din țări de gheață urgisită”}, spațiu rece, ceea ce explică aspirația ei spre soare și lumină, dar și mișcarea de transhumanță care ocazionează popasul în ținutul rigăi.

Membrii cuplului fac parte din regnuri diferite și, de aceea, nu pot comunica în plan real. Întâlnirea lor se realizează în visul fetei, la fel ca în \textbfit{„Luceafărul”}. Riga este cel care rostește de trei ori descântecul de dragoste și, de tot atâtea ori, lapona îl respinge. Povestea propriu-zisă se dovedește a fi fantastică, ca și în poemul eminescian, doar că rolurile sunt inversate. În dialogul lor, formulele de adresare sugerează familiaritate, afecțiune blândă: repetiția \textit{„Enigel, Enigel”}, epitetul \textit{„rigă blând”}.

În prima chemare-descântec, cu rezonanțe de incantație magică, Crypto își îmbie aleasa cu \textit{„dulceață”} și cu \textit{„fulgi”}, elemente ale existenței sale vegetative, dar care aici capătă conotații erotice. Darul lui este refuzat categoric de Enigel: \textit{„Eu mă duc să culeg/Fragii fragezi mai la vale”}. Refuzul laponei îl pune într-o situație dilematică, dar opțiunea lui e fermă și merge până la sacrificiul de sine, în a doua chemare: \textit{„Dacă pleci să culegi/Începi, rogu-te, cu mine”}.

Al doilea refuz este susținut de enumerarea atributelor lui Crypto: \textit{„blând”}, \textit{„plăpând”}, \textit{necopt -- „Lasă. Așteaptă de te coace”}. Opoziția \textit{„copt”} -- \textit{„necopt”}, reluată în al treilea refuz prin antiteza \textit{soare-umbră}, pune în evidență incompatibilitatea lor. Imaginii de fragilitate a lui Crypto lapona îi opune aspirația ei spre absolut. Soarele este simbolul existenței spirituale, al împlinirii umane, în antiteză cu \textit{„umbra”}, simbol al existenței instinctuale, sterile, vegetative.

Pentru a-și continua drumul către soare și cunoaștere, lapona refuză descântecul rigăi, deși regretă și plânge. Descântecul se întoarce în mod brutal asupra celui care l-a rostit și-l distruge. Făptura firavă este distrusă de propriul vis, cade victimă neputinței și îndrăznelii de a-și depăși limitele.

Finalul este trist. Riga Crypto se transformă într-o ciupercă otrăvitoare, obligat să rămână alături de făpturi asemenea lui, \textit{„Laurul-Balaurul”} și \textit{„măsălarița-mireasă”}. Încercarea ființei inferioare de a-și depăși limitele este pedepsită cu nebunia.

\marginnote{figuri semantice/ tropi}[1.3cm]
Soarele, simbolul spiritului, este imaginat în poem prin metaforele \textit{„roata albă”} (perfecțiunea geometrică) și \textit{„aprins inel”} (simbolul nunții), în antiteză cu \textit{„umbra”}, iar metafora \textit{„sufletul-fântână”} sugerează puritatea, setea de cunoaștere, veșnicia, fiind în antiteză cu \textit{„carnea”} (trupul, instinctele). Spiritul și sufletul sunt atribute ale ființei raționale, înțelepte. Făpturile inferioare care aspiră să dobândească spiritualitate sunt distruse de propriul vis, așa cum i se întâmplă lui Crypto, care „înnebunește” și se transformă în ciupercă otrăvitoare.

Trei mituri fundamentale de origine greacă sunt valorificate în opera poetului: al soarelui (absolutul), al nunții și al oglinzii.

Drumul spre sud al laponei are semnificația unui drum inițiatic, iar popasul în ținutul rigăi este o probă, trecută prin respingerea nunții pe o treaptă inferioară.
\marginnote{semnificații}[0.3cm]
Drumul trece prin cercul Venerii (iubirea ce reduce omul la ipostază de ființă instinctuală), apoi sufletul trebuie să mai urce o treaptă, cercul lui Mercur, mai pur, al intelectului, al cunoașterii raționale. Inițierea completă are loc prin adevărata \textit{nuntă} a trupului și spiritului cu însuși focarul vieții, prin trecerea omului în cercul Soarelui (cunoașterea absolută). Aspirația solară a laponei sugerează faptul că, în momentul întâlnirii cu riga Crypto, aceasta se află pe treapta lui Mercur, fără ca ea să fi trăit experiența iubirii. Chemările lui Crypto o atrag spre cercul Venerii. Ea trăiește iubirea ca experiență inițiatică, dar alege să-și urmeze drumul spre Soare (cunoașterea absolută).

\marginnote{figuri de stil}[0.8cm]
Sub raport stilistic, prezența inversiunilor (\textit{„zice-l-aș”}) și a vocativelor în prima parte a baladei evidențiază oralitatea textului. În portretizarea celor două personaje simbolice sunt utilizate epitetul și antiteza: Crypto este \textit{„sterp și nărăvaș”}, \textit{„rigă spân”}; lapona e \textit{„mică, liniștită”} și \textit{„prea-cuminte”}. Ambiguitatea este produsă de metaforele insolite: \textit{„Că sufletul nu e fântână/}[...]\textit{/Pahar e gândul, cu otravă”}.

\marginnote{prozodie}[0.3cm]
Alcătuirea prozodică pare destul de riguroasă inițial: catrene cu rimă încrucișată și măsură predominantă de 8 -- 9 silabe.


\subsection{Concluzie}

Poemul \operatitle\ impune o viziune modernă. Interpretarea dată de însuși \operaauthor\ poemului, \textit{„un \textbf{Luceafăr} întors”}, relevă asemănarea cu problematica capodoperei lui Mihai Eminescu, dar poemul modern este totuși \textit{„un Luceafăr cu rolurile inversate și într-un decor de o nebănuită noutate”}, cum remarcă Nicolae Manolescu.


\chapter{Eseu despre particularitățile de construcție a personajului principal dintr-un roman interbelic}
% Commands
\renewcommand{\operatitle}{\textbfit{„Riga Crypto și lapona Enigel”}} % title of the text
\renewcommand{\operaauthor}{Ion Barbu} % author of the text


% Beginning of text
\subsection{Context}

Publicată în 1924, integrată apoi în volumul \textbfit{„Joc secund”}, balada \operatitle\ face parte din a doua etapă de creație barbiană, numită baladic-orientală, dar anunță dezvoltarea ulterioară a poeziei lui \operaauthor.


\section{Evidențierea trăsăturilor care fac posibilă încadrarea poeziei studiate într-o tipologie, într-un curent cultural/literar, într-o orientare tematică}

\marginnote{poem alegoric}[1.2cm]
\operatitle\ este subintitulată \textit{„Baladă”}, începe ca un cântec bătrânesc de nuntă, dar se realizează în viziune modernă, ca un amplu poem de cunoaștere și poem alegoric, o poveste de iubire din lumea vegetală. Autorul păstrează din specia tradițională schema epică și personajele antagonice, dar evenimentele narate sunt de natură fantastică (dialogul în vis dintre rigă și laponă) și alegorică. Scenariul epic este dublat de caracterul dramatic și de \textit{„lirismul de măști”}, personajele având semnificații simbolice multiple (materia și spiritul etc.).

Poemul se încadrează modernismului interbelic prin intelectualizarea emoției, imaginar poetic inedit, ambiguitate, metafore surprinzătoare și cuvinte cu sonorități neobișnuite, înnoiri prozodice.


\section{Prezentarea imaginilor/ideilor poetice, relevante pentru te\-ma și viziunea despre lume din textul studiat}

Tema poeziei o reprezintă iubirea ca modalitate de cunoaștere a lumii. Fiind „un \textbfit{Luceafăr} întors”, poemul prezintă drama cunoașterii și a incompatibilității dintre două lumi (regnuri).

Titlul baladei trimite cu gândul la marile povești de dragoste din literatura universală, \textbfit{„Romeo și Julieta”}, \textbfit{„Tristan și Isolda”}. Însă la \operaauthor, membrii cuplului sunt antagonici (fac parte din regnuri diferite). Sunt personaje romantice cu trăsături excepționale, dar negative în raport cu norma comună (Crypto e \textit{„sterp”} și \textit{„nărăvaș/Că nu voia să înflorească”}, iar Enigel este \textit{„prea-cuminte”}).

\marginnote{semnificația titlului}[-0.3cm]
Numele Crypto are dublă semnificație: cel tăinuit, \textit{„inimă ascunsă”}, provenind din adjectivul \textit{„criptic”}, (\textit{„ascuns”}, \textit{„tăinuit”}), dar sugerează, în egală măsură, apartenența sa la familia ciupercilor, numele științific \textit{„criptogame”}. Personajul este rege (rigă) al făpturilor inferioare, din regnul vegetal. Numele cu sonoritate nordică Enigel sugerează originea laponei (de la pol) și trimite probabil la semnificația cuvântului din limba suedeză, \textit{„înger”} (care provine din latinescul \textit{„angelus”}).


\section{Ilustrarea elementelor de compoziție și de limbaj ale textului poetic studiat, semnificative pentru tema și viziunea despre lume {\footnotesize\normalfont(de exemplu: imaginar poetic, titlu, incipit, relații de opoziție și de simetrie, motiv poetic, laitmotiv, figuri semantice/tropi, elemente de prozodie etc.)}}

\marginnote{compoziție}[0.3cm]
La nivel formal, poezia este alcătuită din două părți, fiecare dintre ele prezentând câte o nuntă: una împlinită, cadru al celeilalte nunți, povestită, ratată, modificată în final prin căsătoria lui Crypto cu măsălarița. Formula compozițională este aceea a povestirii în ramă.

Prologul conturează în puține imagini atmosfera de la finalul unei nunți trăite. Primele patru strofe constituie rama viitoarei povești și reprezintă dialogul menestrelului cu \textit{„nuntașul fruntaș”}.

Partea a doua prezintă povestea de iubire neîmplinită dintre Enigel și riga Crypto. Nunta povestită cuprinde mai multe tablouri poetice: portretul și împărăția rigăi Crypto (strofele 5 -- 7), portretul, locurile natale și oprirea din drum a laponei Enigel (strofele 8 -- 9), întâlnirea dintre cei doi (strofa 10), cele trei chemări ale rigăi și primele două refuzuri ale laponei (strofele 11 -- 15), răspunsul laponei și refuzul categoric cu relevarea relației dintre simbolul solar și propria condiție (strofele 16 -- 20), încheierea întâlnirii (strofele 21 -- 22), pedepsirea rigăi în finalul baladei (strofele 23 -- 27). Modurile de expunere sunt, în ordine: descrierea, dialogul și narațiunea.

În expozițiune, sunt prezentate în antiteză portretele membrilor cuplului și locurile lor natale, deosebirile dintre ei generând intriga.

Riga Crypto, \textit{„inimă ascunsă”}, este craiul bureților, căruia dragostea pentru Enigel, \textit{„laponă mică, liniștită”}, îi este fatală. Singura lor asemănare este statutul superior în interiorul propriei lumi: el este rigă al plantelor inferioare, care nu înfloresc, iar păstorița care își conduce turmele de reni spre sud este o stăpână a regnului animal, în ipostaza de ființă rațională, omul -- \textit{„fiară bătrână”}.

\marginnote{imaginar poetic}[0.3cm]
Spațiul definitoriu al existenței, pentru Crypto, este umezeala perpetuă și impură, în timp ce lapona vine \textit{„din țări de gheață urgisită”}, spațiu rece, ceea ce explică aspirația ei spre soare și lumină, dar și mișcarea de transhumanță care ocazionează popasul în ținutul rigăi.

Membrii cuplului fac parte din regnuri diferite și, de aceea, nu pot comunica în plan real. Întâlnirea lor se realizează în visul fetei, la fel ca în \textbfit{„Luceafărul”}. Riga este cel care rostește de trei ori descântecul de dragoste și, de tot atâtea ori, lapona îl respinge. Povestea propriu-zisă se dovedește a fi fantastică, ca și în poemul eminescian, doar că rolurile sunt inversate. În dialogul lor, formulele de adresare sugerează familiaritate, afecțiune blândă: repetiția \textit{„Enigel, Enigel”}, epitetul \textit{„rigă blând”}.

În prima chemare-descântec, cu rezonanțe de incantație magică, Crypto își îmbie aleasa cu \textit{„dulceață”} și cu \textit{„fulgi”}, elemente ale existenței sale vegetative, dar care aici capătă conotații erotice. Darul lui este refuzat categoric de Enigel: \textit{„Eu mă duc să culeg/Fragii fragezi mai la vale”}. Refuzul laponei îl pune într-o situație dilematică, dar opțiunea lui e fermă și merge până la sacrificiul de sine, în a doua chemare: \textit{„Dacă pleci să culegi/Începi, rogu-te, cu mine”}.

Al doilea refuz este susținut de enumerarea atributelor lui Crypto: \textit{„blând”}, \textit{„plăpând”}, \textit{necopt -- „Lasă. Așteaptă de te coace”}. Opoziția \textit{„copt”} -- \textit{„necopt”}, reluată în al treilea refuz prin antiteza \textit{soare-umbră}, pune în evidență incompatibilitatea lor. Imaginii de fragilitate a lui Crypto lapona îi opune aspirația ei spre absolut. Soarele este simbolul existenței spirituale, al împlinirii umane, în antiteză cu \textit{„umbra”}, simbol al existenței instinctuale, sterile, vegetative.

Pentru a-și continua drumul către soare și cunoaștere, lapona refuză descântecul rigăi, deși regretă și plânge. Descântecul se întoarce în mod brutal asupra celui care l-a rostit și-l distruge. Făptura firavă este distrusă de propriul vis, cade victimă neputinței și îndrăznelii de a-și depăși limitele.

Finalul este trist. Riga Crypto se transformă într-o ciupercă otrăvitoare, obligat să rămână alături de făpturi asemenea lui, \textit{„Laurul-Balaurul”} și \textit{„măsălarița-mireasă”}. Încercarea ființei inferioare de a-și depăși limitele este pedepsită cu nebunia.

\marginnote{figuri semantice/ tropi}[1.3cm]
Soarele, simbolul spiritului, este imaginat în poem prin metaforele \textit{„roata albă”} (perfecțiunea geometrică) și \textit{„aprins inel”} (simbolul nunții), în antiteză cu \textit{„umbra”}, iar metafora \textit{„sufletul-fântână”} sugerează puritatea, setea de cunoaștere, veșnicia, fiind în antiteză cu \textit{„carnea”} (trupul, instinctele). Spiritul și sufletul sunt atribute ale ființei raționale, înțelepte. Făpturile inferioare care aspiră să dobândească spiritualitate sunt distruse de propriul vis, așa cum i se întâmplă lui Crypto, care „înnebunește” și se transformă în ciupercă otrăvitoare.

Trei mituri fundamentale de origine greacă sunt valorificate în opera poetului: al soarelui (absolutul), al nunții și al oglinzii.

Drumul spre sud al laponei are semnificația unui drum inițiatic, iar popasul în ținutul rigăi este o probă, trecută prin respingerea nunții pe o treaptă inferioară.
\marginnote{semnificații}[0.3cm]
Drumul trece prin cercul Venerii (iubirea ce reduce omul la ipostază de ființă instinctuală), apoi sufletul trebuie să mai urce o treaptă, cercul lui Mercur, mai pur, al intelectului, al cunoașterii raționale. Inițierea completă are loc prin adevărata \textit{nuntă} a trupului și spiritului cu însuși focarul vieții, prin trecerea omului în cercul Soarelui (cunoașterea absolută). Aspirația solară a laponei sugerează faptul că, în momentul întâlnirii cu riga Crypto, aceasta se află pe treapta lui Mercur, fără ca ea să fi trăit experiența iubirii. Chemările lui Crypto o atrag spre cercul Venerii. Ea trăiește iubirea ca experiență inițiatică, dar alege să-și urmeze drumul spre Soare (cunoașterea absolută).

\marginnote{figuri de stil}[0.8cm]
Sub raport stilistic, prezența inversiunilor (\textit{„zice-l-aș”}) și a vocativelor în prima parte a baladei evidențiază oralitatea textului. În portretizarea celor două personaje simbolice sunt utilizate epitetul și antiteza: Crypto este \textit{„sterp și nărăvaș”}, \textit{„rigă spân”}; lapona e \textit{„mică, liniștită”} și \textit{„prea-cuminte”}. Ambiguitatea este produsă de metaforele insolite: \textit{„Că sufletul nu e fântână/}[...]\textit{/Pahar e gândul, cu otravă”}.

\marginnote{prozodie}[0.3cm]
Alcătuirea prozodică pare destul de riguroasă inițial: catrene cu rimă încrucișată și măsură predominantă de 8 -- 9 silabe.


\subsection{Concluzie}

Poemul \operatitle\ impune o viziune modernă. Interpretarea dată de însuși \operaauthor\ poemului, \textit{„un \textbf{Luceafăr} întors”}, relevă asemănarea cu problematica capodoperei lui Mihai Eminescu, dar poemul modern este totuși \textit{„un Luceafăr cu rolurile inversate și într-un decor de o nebănuită noutate”}, cum remarcă Nicolae Manolescu.



% enigma otiliei
\chapter{Eseu privind tema și viziunea despre lume într-un roman realist-balzacian studiat}
% Commands
\renewcommand{\operatitle}{\textbfit{„Riga Crypto și lapona Enigel”}} % title of the text
\renewcommand{\operaauthor}{Ion Barbu} % author of the text


% Beginning of text
\subsection{Context}

Publicată în 1924, integrată apoi în volumul \textbfit{„Joc secund”}, balada \operatitle\ face parte din a doua etapă de creație barbiană, numită baladic-orientală, dar anunță dezvoltarea ulterioară a poeziei lui \operaauthor.


\section{Evidențierea trăsăturilor care fac posibilă încadrarea poeziei studiate într-o tipologie, într-un curent cultural/literar, într-o orientare tematică}

\marginnote{poem alegoric}[1.2cm]
\operatitle\ este subintitulată \textit{„Baladă”}, începe ca un cântec bătrânesc de nuntă, dar se realizează în viziune modernă, ca un amplu poem de cunoaștere și poem alegoric, o poveste de iubire din lumea vegetală. Autorul păstrează din specia tradițională schema epică și personajele antagonice, dar evenimentele narate sunt de natură fantastică (dialogul în vis dintre rigă și laponă) și alegorică. Scenariul epic este dublat de caracterul dramatic și de \textit{„lirismul de măști”}, personajele având semnificații simbolice multiple (materia și spiritul etc.).

Poemul se încadrează modernismului interbelic prin intelectualizarea emoției, imaginar poetic inedit, ambiguitate, metafore surprinzătoare și cuvinte cu sonorități neobișnuite, înnoiri prozodice.


\section{Prezentarea imaginilor/ideilor poetice, relevante pentru te\-ma și viziunea despre lume din textul studiat}

Tema poeziei o reprezintă iubirea ca modalitate de cunoaștere a lumii. Fiind „un \textbfit{Luceafăr} întors”, poemul prezintă drama cunoașterii și a incompatibilității dintre două lumi (regnuri).

Titlul baladei trimite cu gândul la marile povești de dragoste din literatura universală, \textbfit{„Romeo și Julieta”}, \textbfit{„Tristan și Isolda”}. Însă la \operaauthor, membrii cuplului sunt antagonici (fac parte din regnuri diferite). Sunt personaje romantice cu trăsături excepționale, dar negative în raport cu norma comună (Crypto e \textit{„sterp”} și \textit{„nărăvaș/Că nu voia să înflorească”}, iar Enigel este \textit{„prea-cuminte”}).

\marginnote{semnificația titlului}[-0.3cm]
Numele Crypto are dublă semnificație: cel tăinuit, \textit{„inimă ascunsă”}, provenind din adjectivul \textit{„criptic”}, (\textit{„ascuns”}, \textit{„tăinuit”}), dar sugerează, în egală măsură, apartenența sa la familia ciupercilor, numele științific \textit{„criptogame”}. Personajul este rege (rigă) al făpturilor inferioare, din regnul vegetal. Numele cu sonoritate nordică Enigel sugerează originea laponei (de la pol) și trimite probabil la semnificația cuvântului din limba suedeză, \textit{„înger”} (care provine din latinescul \textit{„angelus”}).


\section{Ilustrarea elementelor de compoziție și de limbaj ale textului poetic studiat, semnificative pentru tema și viziunea despre lume {\footnotesize\normalfont(de exemplu: imaginar poetic, titlu, incipit, relații de opoziție și de simetrie, motiv poetic, laitmotiv, figuri semantice/tropi, elemente de prozodie etc.)}}

\marginnote{compoziție}[0.3cm]
La nivel formal, poezia este alcătuită din două părți, fiecare dintre ele prezentând câte o nuntă: una împlinită, cadru al celeilalte nunți, povestită, ratată, modificată în final prin căsătoria lui Crypto cu măsălarița. Formula compozițională este aceea a povestirii în ramă.

Prologul conturează în puține imagini atmosfera de la finalul unei nunți trăite. Primele patru strofe constituie rama viitoarei povești și reprezintă dialogul menestrelului cu \textit{„nuntașul fruntaș”}.

Partea a doua prezintă povestea de iubire neîmplinită dintre Enigel și riga Crypto. Nunta povestită cuprinde mai multe tablouri poetice: portretul și împărăția rigăi Crypto (strofele 5 -- 7), portretul, locurile natale și oprirea din drum a laponei Enigel (strofele 8 -- 9), întâlnirea dintre cei doi (strofa 10), cele trei chemări ale rigăi și primele două refuzuri ale laponei (strofele 11 -- 15), răspunsul laponei și refuzul categoric cu relevarea relației dintre simbolul solar și propria condiție (strofele 16 -- 20), încheierea întâlnirii (strofele 21 -- 22), pedepsirea rigăi în finalul baladei (strofele 23 -- 27). Modurile de expunere sunt, în ordine: descrierea, dialogul și narațiunea.

În expozițiune, sunt prezentate în antiteză portretele membrilor cuplului și locurile lor natale, deosebirile dintre ei generând intriga.

Riga Crypto, \textit{„inimă ascunsă”}, este craiul bureților, căruia dragostea pentru Enigel, \textit{„laponă mică, liniștită”}, îi este fatală. Singura lor asemănare este statutul superior în interiorul propriei lumi: el este rigă al plantelor inferioare, care nu înfloresc, iar păstorița care își conduce turmele de reni spre sud este o stăpână a regnului animal, în ipostaza de ființă rațională, omul -- \textit{„fiară bătrână”}.

\marginnote{imaginar poetic}[0.3cm]
Spațiul definitoriu al existenței, pentru Crypto, este umezeala perpetuă și impură, în timp ce lapona vine \textit{„din țări de gheață urgisită”}, spațiu rece, ceea ce explică aspirația ei spre soare și lumină, dar și mișcarea de transhumanță care ocazionează popasul în ținutul rigăi.

Membrii cuplului fac parte din regnuri diferite și, de aceea, nu pot comunica în plan real. Întâlnirea lor se realizează în visul fetei, la fel ca în \textbfit{„Luceafărul”}. Riga este cel care rostește de trei ori descântecul de dragoste și, de tot atâtea ori, lapona îl respinge. Povestea propriu-zisă se dovedește a fi fantastică, ca și în poemul eminescian, doar că rolurile sunt inversate. În dialogul lor, formulele de adresare sugerează familiaritate, afecțiune blândă: repetiția \textit{„Enigel, Enigel”}, epitetul \textit{„rigă blând”}.

În prima chemare-descântec, cu rezonanțe de incantație magică, Crypto își îmbie aleasa cu \textit{„dulceață”} și cu \textit{„fulgi”}, elemente ale existenței sale vegetative, dar care aici capătă conotații erotice. Darul lui este refuzat categoric de Enigel: \textit{„Eu mă duc să culeg/Fragii fragezi mai la vale”}. Refuzul laponei îl pune într-o situație dilematică, dar opțiunea lui e fermă și merge până la sacrificiul de sine, în a doua chemare: \textit{„Dacă pleci să culegi/Începi, rogu-te, cu mine”}.

Al doilea refuz este susținut de enumerarea atributelor lui Crypto: \textit{„blând”}, \textit{„plăpând”}, \textit{necopt -- „Lasă. Așteaptă de te coace”}. Opoziția \textit{„copt”} -- \textit{„necopt”}, reluată în al treilea refuz prin antiteza \textit{soare-umbră}, pune în evidență incompatibilitatea lor. Imaginii de fragilitate a lui Crypto lapona îi opune aspirația ei spre absolut. Soarele este simbolul existenței spirituale, al împlinirii umane, în antiteză cu \textit{„umbra”}, simbol al existenței instinctuale, sterile, vegetative.

Pentru a-și continua drumul către soare și cunoaștere, lapona refuză descântecul rigăi, deși regretă și plânge. Descântecul se întoarce în mod brutal asupra celui care l-a rostit și-l distruge. Făptura firavă este distrusă de propriul vis, cade victimă neputinței și îndrăznelii de a-și depăși limitele.

Finalul este trist. Riga Crypto se transformă într-o ciupercă otrăvitoare, obligat să rămână alături de făpturi asemenea lui, \textit{„Laurul-Balaurul”} și \textit{„măsălarița-mireasă”}. Încercarea ființei inferioare de a-și depăși limitele este pedepsită cu nebunia.

\marginnote{figuri semantice/ tropi}[1.3cm]
Soarele, simbolul spiritului, este imaginat în poem prin metaforele \textit{„roata albă”} (perfecțiunea geometrică) și \textit{„aprins inel”} (simbolul nunții), în antiteză cu \textit{„umbra”}, iar metafora \textit{„sufletul-fântână”} sugerează puritatea, setea de cunoaștere, veșnicia, fiind în antiteză cu \textit{„carnea”} (trupul, instinctele). Spiritul și sufletul sunt atribute ale ființei raționale, înțelepte. Făpturile inferioare care aspiră să dobândească spiritualitate sunt distruse de propriul vis, așa cum i se întâmplă lui Crypto, care „înnebunește” și se transformă în ciupercă otrăvitoare.

Trei mituri fundamentale de origine greacă sunt valorificate în opera poetului: al soarelui (absolutul), al nunții și al oglinzii.

Drumul spre sud al laponei are semnificația unui drum inițiatic, iar popasul în ținutul rigăi este o probă, trecută prin respingerea nunții pe o treaptă inferioară.
\marginnote{semnificații}[0.3cm]
Drumul trece prin cercul Venerii (iubirea ce reduce omul la ipostază de ființă instinctuală), apoi sufletul trebuie să mai urce o treaptă, cercul lui Mercur, mai pur, al intelectului, al cunoașterii raționale. Inițierea completă are loc prin adevărata \textit{nuntă} a trupului și spiritului cu însuși focarul vieții, prin trecerea omului în cercul Soarelui (cunoașterea absolută). Aspirația solară a laponei sugerează faptul că, în momentul întâlnirii cu riga Crypto, aceasta se află pe treapta lui Mercur, fără ca ea să fi trăit experiența iubirii. Chemările lui Crypto o atrag spre cercul Venerii. Ea trăiește iubirea ca experiență inițiatică, dar alege să-și urmeze drumul spre Soare (cunoașterea absolută).

\marginnote{figuri de stil}[0.8cm]
Sub raport stilistic, prezența inversiunilor (\textit{„zice-l-aș”}) și a vocativelor în prima parte a baladei evidențiază oralitatea textului. În portretizarea celor două personaje simbolice sunt utilizate epitetul și antiteza: Crypto este \textit{„sterp și nărăvaș”}, \textit{„rigă spân”}; lapona e \textit{„mică, liniștită”} și \textit{„prea-cuminte”}. Ambiguitatea este produsă de metaforele insolite: \textit{„Că sufletul nu e fântână/}[...]\textit{/Pahar e gândul, cu otravă”}.

\marginnote{prozodie}[0.3cm]
Alcătuirea prozodică pare destul de riguroasă inițial: catrene cu rimă încrucișată și măsură predominantă de 8 -- 9 silabe.


\subsection{Concluzie}

Poemul \operatitle\ impune o viziune modernă. Interpretarea dată de însuși \operaauthor\ poemului, \textit{„un \textbf{Luceafăr} întors”}, relevă asemănarea cu problematica capodoperei lui Mihai Eminescu, dar poemul modern este totuși \textit{„un Luceafăr cu rolurile inversate și într-un decor de o nebănuită noutate”}, cum remarcă Nicolae Manolescu.


\chapter{Eseu despre particularitățile de construcție a personajului principal dintr-un roman realist-balzacian}
% Commands
\renewcommand{\operatitle}{\textbfit{„Riga Crypto și lapona Enigel”}} % title of the text
\renewcommand{\operaauthor}{Ion Barbu} % author of the text


% Beginning of text
\subsection{Context}

Publicată în 1924, integrată apoi în volumul \textbfit{„Joc secund”}, balada \operatitle\ face parte din a doua etapă de creație barbiană, numită baladic-orientală, dar anunță dezvoltarea ulterioară a poeziei lui \operaauthor.


\section{Evidențierea trăsăturilor care fac posibilă încadrarea poeziei studiate într-o tipologie, într-un curent cultural/literar, într-o orientare tematică}

\marginnote{poem alegoric}[1.2cm]
\operatitle\ este subintitulată \textit{„Baladă”}, începe ca un cântec bătrânesc de nuntă, dar se realizează în viziune modernă, ca un amplu poem de cunoaștere și poem alegoric, o poveste de iubire din lumea vegetală. Autorul păstrează din specia tradițională schema epică și personajele antagonice, dar evenimentele narate sunt de natură fantastică (dialogul în vis dintre rigă și laponă) și alegorică. Scenariul epic este dublat de caracterul dramatic și de \textit{„lirismul de măști”}, personajele având semnificații simbolice multiple (materia și spiritul etc.).

Poemul se încadrează modernismului interbelic prin intelectualizarea emoției, imaginar poetic inedit, ambiguitate, metafore surprinzătoare și cuvinte cu sonorități neobișnuite, înnoiri prozodice.


\section{Prezentarea imaginilor/ideilor poetice, relevante pentru te\-ma și viziunea despre lume din textul studiat}

Tema poeziei o reprezintă iubirea ca modalitate de cunoaștere a lumii. Fiind „un \textbfit{Luceafăr} întors”, poemul prezintă drama cunoașterii și a incompatibilității dintre două lumi (regnuri).

Titlul baladei trimite cu gândul la marile povești de dragoste din literatura universală, \textbfit{„Romeo și Julieta”}, \textbfit{„Tristan și Isolda”}. Însă la \operaauthor, membrii cuplului sunt antagonici (fac parte din regnuri diferite). Sunt personaje romantice cu trăsături excepționale, dar negative în raport cu norma comună (Crypto e \textit{„sterp”} și \textit{„nărăvaș/Că nu voia să înflorească”}, iar Enigel este \textit{„prea-cuminte”}).

\marginnote{semnificația titlului}[-0.3cm]
Numele Crypto are dublă semnificație: cel tăinuit, \textit{„inimă ascunsă”}, provenind din adjectivul \textit{„criptic”}, (\textit{„ascuns”}, \textit{„tăinuit”}), dar sugerează, în egală măsură, apartenența sa la familia ciupercilor, numele științific \textit{„criptogame”}. Personajul este rege (rigă) al făpturilor inferioare, din regnul vegetal. Numele cu sonoritate nordică Enigel sugerează originea laponei (de la pol) și trimite probabil la semnificația cuvântului din limba suedeză, \textit{„înger”} (care provine din latinescul \textit{„angelus”}).


\section{Ilustrarea elementelor de compoziție și de limbaj ale textului poetic studiat, semnificative pentru tema și viziunea despre lume {\footnotesize\normalfont(de exemplu: imaginar poetic, titlu, incipit, relații de opoziție și de simetrie, motiv poetic, laitmotiv, figuri semantice/tropi, elemente de prozodie etc.)}}

\marginnote{compoziție}[0.3cm]
La nivel formal, poezia este alcătuită din două părți, fiecare dintre ele prezentând câte o nuntă: una împlinită, cadru al celeilalte nunți, povestită, ratată, modificată în final prin căsătoria lui Crypto cu măsălarița. Formula compozițională este aceea a povestirii în ramă.

Prologul conturează în puține imagini atmosfera de la finalul unei nunți trăite. Primele patru strofe constituie rama viitoarei povești și reprezintă dialogul menestrelului cu \textit{„nuntașul fruntaș”}.

Partea a doua prezintă povestea de iubire neîmplinită dintre Enigel și riga Crypto. Nunta povestită cuprinde mai multe tablouri poetice: portretul și împărăția rigăi Crypto (strofele 5 -- 7), portretul, locurile natale și oprirea din drum a laponei Enigel (strofele 8 -- 9), întâlnirea dintre cei doi (strofa 10), cele trei chemări ale rigăi și primele două refuzuri ale laponei (strofele 11 -- 15), răspunsul laponei și refuzul categoric cu relevarea relației dintre simbolul solar și propria condiție (strofele 16 -- 20), încheierea întâlnirii (strofele 21 -- 22), pedepsirea rigăi în finalul baladei (strofele 23 -- 27). Modurile de expunere sunt, în ordine: descrierea, dialogul și narațiunea.

În expozițiune, sunt prezentate în antiteză portretele membrilor cuplului și locurile lor natale, deosebirile dintre ei generând intriga.

Riga Crypto, \textit{„inimă ascunsă”}, este craiul bureților, căruia dragostea pentru Enigel, \textit{„laponă mică, liniștită”}, îi este fatală. Singura lor asemănare este statutul superior în interiorul propriei lumi: el este rigă al plantelor inferioare, care nu înfloresc, iar păstorița care își conduce turmele de reni spre sud este o stăpână a regnului animal, în ipostaza de ființă rațională, omul -- \textit{„fiară bătrână”}.

\marginnote{imaginar poetic}[0.3cm]
Spațiul definitoriu al existenței, pentru Crypto, este umezeala perpetuă și impură, în timp ce lapona vine \textit{„din țări de gheață urgisită”}, spațiu rece, ceea ce explică aspirația ei spre soare și lumină, dar și mișcarea de transhumanță care ocazionează popasul în ținutul rigăi.

Membrii cuplului fac parte din regnuri diferite și, de aceea, nu pot comunica în plan real. Întâlnirea lor se realizează în visul fetei, la fel ca în \textbfit{„Luceafărul”}. Riga este cel care rostește de trei ori descântecul de dragoste și, de tot atâtea ori, lapona îl respinge. Povestea propriu-zisă se dovedește a fi fantastică, ca și în poemul eminescian, doar că rolurile sunt inversate. În dialogul lor, formulele de adresare sugerează familiaritate, afecțiune blândă: repetiția \textit{„Enigel, Enigel”}, epitetul \textit{„rigă blând”}.

În prima chemare-descântec, cu rezonanțe de incantație magică, Crypto își îmbie aleasa cu \textit{„dulceață”} și cu \textit{„fulgi”}, elemente ale existenței sale vegetative, dar care aici capătă conotații erotice. Darul lui este refuzat categoric de Enigel: \textit{„Eu mă duc să culeg/Fragii fragezi mai la vale”}. Refuzul laponei îl pune într-o situație dilematică, dar opțiunea lui e fermă și merge până la sacrificiul de sine, în a doua chemare: \textit{„Dacă pleci să culegi/Începi, rogu-te, cu mine”}.

Al doilea refuz este susținut de enumerarea atributelor lui Crypto: \textit{„blând”}, \textit{„plăpând”}, \textit{necopt -- „Lasă. Așteaptă de te coace”}. Opoziția \textit{„copt”} -- \textit{„necopt”}, reluată în al treilea refuz prin antiteza \textit{soare-umbră}, pune în evidență incompatibilitatea lor. Imaginii de fragilitate a lui Crypto lapona îi opune aspirația ei spre absolut. Soarele este simbolul existenței spirituale, al împlinirii umane, în antiteză cu \textit{„umbra”}, simbol al existenței instinctuale, sterile, vegetative.

Pentru a-și continua drumul către soare și cunoaștere, lapona refuză descântecul rigăi, deși regretă și plânge. Descântecul se întoarce în mod brutal asupra celui care l-a rostit și-l distruge. Făptura firavă este distrusă de propriul vis, cade victimă neputinței și îndrăznelii de a-și depăși limitele.

Finalul este trist. Riga Crypto se transformă într-o ciupercă otrăvitoare, obligat să rămână alături de făpturi asemenea lui, \textit{„Laurul-Balaurul”} și \textit{„măsălarița-mireasă”}. Încercarea ființei inferioare de a-și depăși limitele este pedepsită cu nebunia.

\marginnote{figuri semantice/ tropi}[1.3cm]
Soarele, simbolul spiritului, este imaginat în poem prin metaforele \textit{„roata albă”} (perfecțiunea geometrică) și \textit{„aprins inel”} (simbolul nunții), în antiteză cu \textit{„umbra”}, iar metafora \textit{„sufletul-fântână”} sugerează puritatea, setea de cunoaștere, veșnicia, fiind în antiteză cu \textit{„carnea”} (trupul, instinctele). Spiritul și sufletul sunt atribute ale ființei raționale, înțelepte. Făpturile inferioare care aspiră să dobândească spiritualitate sunt distruse de propriul vis, așa cum i se întâmplă lui Crypto, care „înnebunește” și se transformă în ciupercă otrăvitoare.

Trei mituri fundamentale de origine greacă sunt valorificate în opera poetului: al soarelui (absolutul), al nunții și al oglinzii.

Drumul spre sud al laponei are semnificația unui drum inițiatic, iar popasul în ținutul rigăi este o probă, trecută prin respingerea nunții pe o treaptă inferioară.
\marginnote{semnificații}[0.3cm]
Drumul trece prin cercul Venerii (iubirea ce reduce omul la ipostază de ființă instinctuală), apoi sufletul trebuie să mai urce o treaptă, cercul lui Mercur, mai pur, al intelectului, al cunoașterii raționale. Inițierea completă are loc prin adevărata \textit{nuntă} a trupului și spiritului cu însuși focarul vieții, prin trecerea omului în cercul Soarelui (cunoașterea absolută). Aspirația solară a laponei sugerează faptul că, în momentul întâlnirii cu riga Crypto, aceasta se află pe treapta lui Mercur, fără ca ea să fi trăit experiența iubirii. Chemările lui Crypto o atrag spre cercul Venerii. Ea trăiește iubirea ca experiență inițiatică, dar alege să-și urmeze drumul spre Soare (cunoașterea absolută).

\marginnote{figuri de stil}[0.8cm]
Sub raport stilistic, prezența inversiunilor (\textit{„zice-l-aș”}) și a vocativelor în prima parte a baladei evidențiază oralitatea textului. În portretizarea celor două personaje simbolice sunt utilizate epitetul și antiteza: Crypto este \textit{„sterp și nărăvaș”}, \textit{„rigă spân”}; lapona e \textit{„mică, liniștită”} și \textit{„prea-cuminte”}. Ambiguitatea este produsă de metaforele insolite: \textit{„Că sufletul nu e fântână/}[...]\textit{/Pahar e gândul, cu otravă”}.

\marginnote{prozodie}[0.3cm]
Alcătuirea prozodică pare destul de riguroasă inițial: catrene cu rimă încrucișată și măsură predominantă de 8 -- 9 silabe.


\subsection{Concluzie}

Poemul \operatitle\ impune o viziune modernă. Interpretarea dată de însuși \operaauthor\ poemului, \textit{„un \textbf{Luceafăr} întors”}, relevă asemănarea cu problematica capodoperei lui Mihai Eminescu, dar poemul modern este totuși \textit{„un Luceafăr cu rolurile inversate și într-un decor de o nebănuită noutate”}, cum remarcă Nicolae Manolescu.



% ultima noapte
\chapter{Eseu cu privire la tema și viziunea despre lume dintr-un roman psihologic studiat}
% Commands
\renewcommand{\operatitle}{\textbfit{„Ultima noapte de dragoste, întâia noapte de război”}} % title of the text
\renewcommand{\operaauthor}{Camil Petrescu} % author of the text


% Beginning of text
\subsection{Context}

\operaauthor\ ilustrează estetica autenticității în studiile teoretice în romanele sale (\operatitle, 1930; \textbfit{„Patul lui Procust”}, 1933).


\section{Încadrarea romanului studiat într-o tipologie, într-un curent cultural/literar, într-o orientare tematică}

\marginnote{tipologie}[1.3cm]
\operatitle\ este un roman modern, psihologic, de tip subiectiv, care ilustrează afirmațiile pe care Camil Petrescu le va face mai târziu în conferința citată. Romanul are drept caracteristici: unicitatea perspectivei narative, timpul prezent și subiectiv, raportul dintre timpul cronologic și timpul psihologic, fluxul conștiinței, memoria afectivă (involuntară), narațiunea la persoana \rom{1}, luciditatea (auto)analizei, anticalofilismul, dar și autenticitatea trăirii.


\section{Ilustrarea temei romanului prin episoade/citate/secvențe comentate}

\marginnote{temă, structură, titlu}[1.8cm]
Textul narativ este structurat în două părți precizate în titlu, care indică temele romanului și, în același timp, cele două experiențe fundamentale de cunoaștere trăite de protagonist: dragostea și războiul. Dacă prima parte reprezintă rememorarea iubirii matrimoniale eșuate dintre Ștefan Gheorghidiu și Ela, partea a doua, construită sub forma jurnalului de campanie al lui Gheorghidiu, urmărește experiența de pe front, în timpul Primului Război Mondial. Prima parte este în întregime ficțională, în timp ce a doua valorifică jurnalul de campanie al autorului, ceea ce conferă autenticitate textului.

Romanul debutează printr-un artificiu compozițional: acțiunea primului capitol, \textit{„La Piatra Craiului, în munte”}, este posterioară întâmplărilor relatate în restul \textit{„Cărții \rom{1}”}. Capitolul scoate în evidență cele două planuri temporale din discursul narativ: timpul narării (prezentul frontului) și timpul narat (trecutul poveștii de iubire). În primăvara lui 1916, în timpul unei concentrări pe Valea Prahovei, Gheorghidiu asistă la popota ofițerilor la o discuție despre dragoste și fidelitate, pornind de la un fapt divers aflat din presă: un bărbat care și-a ucis soția infidelă a fost achitat la tribunal. Această discuție declanșează memoria afectivă a protagonistului, trezindu-i amintirile legate de cei doi ani și jumătate de căsnicie cu Ela.

Întocmai ca la Proust, un eveniment exterior declanșează rememorarea unor întâmplări sau stări trăite într-un timp trecut.


\section{Prezentarea elementelor de structură și de compoziție ale textului narativ, semnificative pentru tema și viziunea despre lume din romanul studiat {\footnotesize\normalfont(de exemplu: acțiune, conflict, relații temporale și spațiale, incipit, final, tehnici narative, perspectivă narativă, registre stilistice, limbajul personajelor etc.)}}

\marginnote{incipit, final deschis}[0.8cm]
Chiar dacă este vorba de un roman modern, în incipit sunt fixate cu precizie realizată coordonatele spațio-temporale: \textit{„În primăvara anului 1916,} [...] \textit{între Bușteni și Predeal.”} În schimb, finalul deschis lasă loc interpretărilor multiple, așa cum se întâmplă în general în proza de analiză psihologică. Astfel, Gheorghidiu, obosit să mai caute certitudini și să se mai îndoiască, se simte detașat de tot ceea ce îl legase de Ela și hotărăște să o părăsească, să-i lase \textit{„tot trecutul.”}

În romanul lui Camil Petrescu, apare conflictul interior, din conștiința personajului narator, Ștefan Gheorghidiu, care trăiește stări și sentimente contradictorii față de soția sa, Ela.
\marginnote{conflict}[0.3cm]
Acest conflict interior este generat de raporturile pe care protagonistul le are cu realitatea înconjurătoare. Principalul motiv al rupturii dintre Ștefan și soția sa este implicarea Elei în lumea mondenă, pe care eroul o disprețuiește. Așadar, conflictul interior se produce din cauza diferenței dintre aspirațiile lui Gheorghidiu și realitatea lumii înconjurătoare.

Conflictul interior este dublat de un conflict exterior generat de relația protagonistului cu societatea, acesta fiind plasat în categoria inadaptaților social.

\marginnote{perspectiva narativă}[0.1cm]
Romanul este scris la persoana \rom{1}, sub forma unei confesiuni a personajului principal, Ștefan Gheorghidiu, care trăiește două experiențe fundamentale: iubirea și războiul. Relatarea la persoana \rom{1} conferă autenticitate și caracter subiectiv textului.

Romanul este alcătuit din două părți și treisprezece capitole cu titluri sugestive.

\marginnote{structura, acțiunea}[0.3cm]
\textit{„Eram însurat de doi ani și jumătate cu o colegă de la Universitate și bănuiam că mă înșală”} este fraza prin care debutează abrupt cel de-al doilea capitol, dar și retrospectiva iubirii dintre Ștefan Gheorghidiu și Ela. Tânărul, pe atunci student la Filosofie, se căsătorește din dragoste cu Ela, studentă la Litere, orfană crescută de o mătușă. Iubirea bărbatului se naște din admirație, din duioșie, dar mai ales din orgoliu, fiindcă Ela era cea mai frumoasă și cea mai populară studentă de la Universitate, iar faptul că era îndrăgostită de Ștefan trezea admirația și invidia colegilor.

După căsătorie, cei doi soți trăiesc modest, dar sunt fericiți. Echilibrul tinerei familii este tulburat de o moștenire pe care Gheorghidiu o primește de la moartea unchiului său, Tache. Ela se implică în discuțiile despre bani, lucru care lui Gheorghidiu îi displace profund: \textit{„Aș fi vrut-o mereu feminină, deasupra acestor discuții vulgare”}. Mai mult, spre deosebire de soțul său, se pare că Ela era atrasă de viața mondenă, la care are acces datorită noului statut social al familiei. Cuplul evoluează spre o inevitabilă criză matrimonială, declanșată cu ocazia excursiei de la Odobești, când Ela pare să-i acorde o atenție exagerată unui anume domn G., \textit{„vag avocat”} și dansator monden. Acesta din urmă, crede personajul-narator, îi va deveni mai târziu amant.

După excursia de la Odobești, relația lor devine o succesiune de separări și împăcări.

\clearpage

Concentrat pe valea Prahovei, unde așteaptă intrarea României în război, Gheorghidiu primește o scrisoare de la Ela prin care aceasta îl cheamă urgent la Câmpulung, unde se mutase pentru a fi mai aproape de el. Soția vrea să-l convingă să treacă o sumă de bani pe numele ei pentru a fi asigurată din punct de vedere financiar în cazul morții lui pe front. Aflând ce-și dorește Ela, Gheorghidiu e convins că ea plănuiește divorțul pentru a rămâne cu domnul G. Întâlnindu-l pe domnul G., protagonistul crede că acesta nu se află întâmplător la Câmpulung și că a venit acolo pentru a fi alături de Ela. Din cauza izbucnirii războiului, Ștefan nu mai are ocazia să verifice dacă soția îl înșală sau nu.

A doua experiență în planul cunoașterii existențiale o reprezintă războiul, care pune în umbră experiența iubirii. Frontul înseamnă haos, mizerie, măsuri absurde, învălmășeală, dezordine, ordine contradictorii. Din cauza informațiilor eronate, artileria română își fixează tunurile asupra propriilor batalioane.

Capitulul \textit{„Ne-a acoperit pământul lui Dumnezeu”} înfățișează imaginea apocaliptică a războiului. Omul mai păstrează doar instinctul de supraviețuire și automatismul, după cum remarcă însuși Gheorghidiu: \textit{„Nu mai e nimic omenesc în noi.”}

Rănit și spitalizat, Gheorghidiu revine acasă, la București, dar se simte detașat de tot ceea ce îl legase de Ela. De aceea, hotărăște să o părăsească și să-i lase \textit{„tot trecutul”}.

Cum sfârșitul lasă loc interpretărilor multiple, se poate considera că romanul are un final deschis.

\marginnote{personajele}[0.3cm]
Personajul-narator, Ștefan Gheorghidiu, reprezintă tipul intelectualului lucid, inadaptatul superior. Filosof, el are impresia că s-a izolat de lumea exterioară, însă, în realitate, evenimentele exterioare sunt filtrate prin conștiința sa. Ela, personajul feminin al romanului, este prezentată doar din perspectiva lui Gheorghidiu. De aceea cititorul nu se poate pronunța asupra fidelității ei și nici nu poate opina dacă e mai degrabă superficială decât spirituală. \textit{„Nu Ela se schimbă, ci felul în care o vede Ștefan”}.

\marginnote{tehnici ale analizei psihologice}[-1.5cm]
Prin introspecție și monolog interior -- tehnici ale analizei psihologice --  Ștefan \hbox{Gheorghidiu} își analizează cu luciditate trăirile, stările și sentimentele.

\marginnote{stilul anticalofil}[-0.1cm]
Stilul anticalofil („împotriva scrisului frumos”) susține autenticitatea limbajului.


\subsection{Concluzie}

\operatitle\ este un roman psihologic modern, având drept caracteristici: unicitatea perspectivei narative, relatarea la persoana \rom{1}, la timpul prezent, subiectivitatea, apelul la memoria afectivă și autenticitatea trăirii.


\chapter{Eseu despre particularitățile de construcție a personajului principal dintr-un roman psihologic}
% Commands
\renewcommand{\operatitle}{\textbfit{„Riga Crypto și lapona Enigel”}} % title of the text
\renewcommand{\operaauthor}{Ion Barbu} % author of the text


% Beginning of text
\subsection{Context}

Publicată în 1924, integrată apoi în volumul \textbfit{„Joc secund”}, balada \operatitle\ face parte din a doua etapă de creație barbiană, numită baladic-orientală, dar anunță dezvoltarea ulterioară a poeziei lui \operaauthor.


\section{Evidențierea trăsăturilor care fac posibilă încadrarea poeziei studiate într-o tipologie, într-un curent cultural/literar, într-o orientare tematică}

\marginnote{poem alegoric}[1.2cm]
\operatitle\ este subintitulată \textit{„Baladă”}, începe ca un cântec bătrânesc de nuntă, dar se realizează în viziune modernă, ca un amplu poem de cunoaștere și poem alegoric, o poveste de iubire din lumea vegetală. Autorul păstrează din specia tradițională schema epică și personajele antagonice, dar evenimentele narate sunt de natură fantastică (dialogul în vis dintre rigă și laponă) și alegorică. Scenariul epic este dublat de caracterul dramatic și de \textit{„lirismul de măști”}, personajele având semnificații simbolice multiple (materia și spiritul etc.).

Poemul se încadrează modernismului interbelic prin intelectualizarea emoției, imaginar poetic inedit, ambiguitate, metafore surprinzătoare și cuvinte cu sonorități neobișnuite, înnoiri prozodice.


\section{Prezentarea imaginilor/ideilor poetice, relevante pentru te\-ma și viziunea despre lume din textul studiat}

Tema poeziei o reprezintă iubirea ca modalitate de cunoaștere a lumii. Fiind „un \textbfit{Luceafăr} întors”, poemul prezintă drama cunoașterii și a incompatibilității dintre două lumi (regnuri).

Titlul baladei trimite cu gândul la marile povești de dragoste din literatura universală, \textbfit{„Romeo și Julieta”}, \textbfit{„Tristan și Isolda”}. Însă la \operaauthor, membrii cuplului sunt antagonici (fac parte din regnuri diferite). Sunt personaje romantice cu trăsături excepționale, dar negative în raport cu norma comună (Crypto e \textit{„sterp”} și \textit{„nărăvaș/Că nu voia să înflorească”}, iar Enigel este \textit{„prea-cuminte”}).

\marginnote{semnificația titlului}[-0.3cm]
Numele Crypto are dublă semnificație: cel tăinuit, \textit{„inimă ascunsă”}, provenind din adjectivul \textit{„criptic”}, (\textit{„ascuns”}, \textit{„tăinuit”}), dar sugerează, în egală măsură, apartenența sa la familia ciupercilor, numele științific \textit{„criptogame”}. Personajul este rege (rigă) al făpturilor inferioare, din regnul vegetal. Numele cu sonoritate nordică Enigel sugerează originea laponei (de la pol) și trimite probabil la semnificația cuvântului din limba suedeză, \textit{„înger”} (care provine din latinescul \textit{„angelus”}).


\section{Ilustrarea elementelor de compoziție și de limbaj ale textului poetic studiat, semnificative pentru tema și viziunea despre lume {\footnotesize\normalfont(de exemplu: imaginar poetic, titlu, incipit, relații de opoziție și de simetrie, motiv poetic, laitmotiv, figuri semantice/tropi, elemente de prozodie etc.)}}

\marginnote{compoziție}[0.3cm]
La nivel formal, poezia este alcătuită din două părți, fiecare dintre ele prezentând câte o nuntă: una împlinită, cadru al celeilalte nunți, povestită, ratată, modificată în final prin căsătoria lui Crypto cu măsălarița. Formula compozițională este aceea a povestirii în ramă.

Prologul conturează în puține imagini atmosfera de la finalul unei nunți trăite. Primele patru strofe constituie rama viitoarei povești și reprezintă dialogul menestrelului cu \textit{„nuntașul fruntaș”}.

Partea a doua prezintă povestea de iubire neîmplinită dintre Enigel și riga Crypto. Nunta povestită cuprinde mai multe tablouri poetice: portretul și împărăția rigăi Crypto (strofele 5 -- 7), portretul, locurile natale și oprirea din drum a laponei Enigel (strofele 8 -- 9), întâlnirea dintre cei doi (strofa 10), cele trei chemări ale rigăi și primele două refuzuri ale laponei (strofele 11 -- 15), răspunsul laponei și refuzul categoric cu relevarea relației dintre simbolul solar și propria condiție (strofele 16 -- 20), încheierea întâlnirii (strofele 21 -- 22), pedepsirea rigăi în finalul baladei (strofele 23 -- 27). Modurile de expunere sunt, în ordine: descrierea, dialogul și narațiunea.

În expozițiune, sunt prezentate în antiteză portretele membrilor cuplului și locurile lor natale, deosebirile dintre ei generând intriga.

Riga Crypto, \textit{„inimă ascunsă”}, este craiul bureților, căruia dragostea pentru Enigel, \textit{„laponă mică, liniștită”}, îi este fatală. Singura lor asemănare este statutul superior în interiorul propriei lumi: el este rigă al plantelor inferioare, care nu înfloresc, iar păstorița care își conduce turmele de reni spre sud este o stăpână a regnului animal, în ipostaza de ființă rațională, omul -- \textit{„fiară bătrână”}.

\marginnote{imaginar poetic}[0.3cm]
Spațiul definitoriu al existenței, pentru Crypto, este umezeala perpetuă și impură, în timp ce lapona vine \textit{„din țări de gheață urgisită”}, spațiu rece, ceea ce explică aspirația ei spre soare și lumină, dar și mișcarea de transhumanță care ocazionează popasul în ținutul rigăi.

Membrii cuplului fac parte din regnuri diferite și, de aceea, nu pot comunica în plan real. Întâlnirea lor se realizează în visul fetei, la fel ca în \textbfit{„Luceafărul”}. Riga este cel care rostește de trei ori descântecul de dragoste și, de tot atâtea ori, lapona îl respinge. Povestea propriu-zisă se dovedește a fi fantastică, ca și în poemul eminescian, doar că rolurile sunt inversate. În dialogul lor, formulele de adresare sugerează familiaritate, afecțiune blândă: repetiția \textit{„Enigel, Enigel”}, epitetul \textit{„rigă blând”}.

În prima chemare-descântec, cu rezonanțe de incantație magică, Crypto își îmbie aleasa cu \textit{„dulceață”} și cu \textit{„fulgi”}, elemente ale existenței sale vegetative, dar care aici capătă conotații erotice. Darul lui este refuzat categoric de Enigel: \textit{„Eu mă duc să culeg/Fragii fragezi mai la vale”}. Refuzul laponei îl pune într-o situație dilematică, dar opțiunea lui e fermă și merge până la sacrificiul de sine, în a doua chemare: \textit{„Dacă pleci să culegi/Începi, rogu-te, cu mine”}.

Al doilea refuz este susținut de enumerarea atributelor lui Crypto: \textit{„blând”}, \textit{„plăpând”}, \textit{necopt -- „Lasă. Așteaptă de te coace”}. Opoziția \textit{„copt”} -- \textit{„necopt”}, reluată în al treilea refuz prin antiteza \textit{soare-umbră}, pune în evidență incompatibilitatea lor. Imaginii de fragilitate a lui Crypto lapona îi opune aspirația ei spre absolut. Soarele este simbolul existenței spirituale, al împlinirii umane, în antiteză cu \textit{„umbra”}, simbol al existenței instinctuale, sterile, vegetative.

Pentru a-și continua drumul către soare și cunoaștere, lapona refuză descântecul rigăi, deși regretă și plânge. Descântecul se întoarce în mod brutal asupra celui care l-a rostit și-l distruge. Făptura firavă este distrusă de propriul vis, cade victimă neputinței și îndrăznelii de a-și depăși limitele.

Finalul este trist. Riga Crypto se transformă într-o ciupercă otrăvitoare, obligat să rămână alături de făpturi asemenea lui, \textit{„Laurul-Balaurul”} și \textit{„măsălarița-mireasă”}. Încercarea ființei inferioare de a-și depăși limitele este pedepsită cu nebunia.

\marginnote{figuri semantice/ tropi}[1.3cm]
Soarele, simbolul spiritului, este imaginat în poem prin metaforele \textit{„roata albă”} (perfecțiunea geometrică) și \textit{„aprins inel”} (simbolul nunții), în antiteză cu \textit{„umbra”}, iar metafora \textit{„sufletul-fântână”} sugerează puritatea, setea de cunoaștere, veșnicia, fiind în antiteză cu \textit{„carnea”} (trupul, instinctele). Spiritul și sufletul sunt atribute ale ființei raționale, înțelepte. Făpturile inferioare care aspiră să dobândească spiritualitate sunt distruse de propriul vis, așa cum i se întâmplă lui Crypto, care „înnebunește” și se transformă în ciupercă otrăvitoare.

Trei mituri fundamentale de origine greacă sunt valorificate în opera poetului: al soarelui (absolutul), al nunții și al oglinzii.

Drumul spre sud al laponei are semnificația unui drum inițiatic, iar popasul în ținutul rigăi este o probă, trecută prin respingerea nunții pe o treaptă inferioară.
\marginnote{semnificații}[0.3cm]
Drumul trece prin cercul Venerii (iubirea ce reduce omul la ipostază de ființă instinctuală), apoi sufletul trebuie să mai urce o treaptă, cercul lui Mercur, mai pur, al intelectului, al cunoașterii raționale. Inițierea completă are loc prin adevărata \textit{nuntă} a trupului și spiritului cu însuși focarul vieții, prin trecerea omului în cercul Soarelui (cunoașterea absolută). Aspirația solară a laponei sugerează faptul că, în momentul întâlnirii cu riga Crypto, aceasta se află pe treapta lui Mercur, fără ca ea să fi trăit experiența iubirii. Chemările lui Crypto o atrag spre cercul Venerii. Ea trăiește iubirea ca experiență inițiatică, dar alege să-și urmeze drumul spre Soare (cunoașterea absolută).

\marginnote{figuri de stil}[0.8cm]
Sub raport stilistic, prezența inversiunilor (\textit{„zice-l-aș”}) și a vocativelor în prima parte a baladei evidențiază oralitatea textului. În portretizarea celor două personaje simbolice sunt utilizate epitetul și antiteza: Crypto este \textit{„sterp și nărăvaș”}, \textit{„rigă spân”}; lapona e \textit{„mică, liniștită”} și \textit{„prea-cuminte”}. Ambiguitatea este produsă de metaforele insolite: \textit{„Că sufletul nu e fântână/}[...]\textit{/Pahar e gândul, cu otravă”}.

\marginnote{prozodie}[0.3cm]
Alcătuirea prozodică pare destul de riguroasă inițial: catrene cu rimă încrucișată și măsură predominantă de 8 -- 9 silabe.


\subsection{Concluzie}

Poemul \operatitle\ impune o viziune modernă. Interpretarea dată de însuși \operaauthor\ poemului, \textit{„un \textbf{Luceafăr} întors”}, relevă asemănarea cu problematica capodoperei lui Mihai Eminescu, dar poemul modern este totuși \textit{„un Luceafăr cu rolurile inversate și într-un decor de o nebănuită noutate”}, cum remarcă Nicolae Manolescu.



% osp
\chapter{Eseu cu privire la tema și viziunea despre lume dintr-o comedie studiată}
% Commands
\renewcommand{\operatitle}{\textbfit{„Riga Crypto și lapona Enigel”}} % title of the text
\renewcommand{\operaauthor}{Ion Barbu} % author of the text


% Beginning of text
\subsection{Context}

Publicată în 1924, integrată apoi în volumul \textbfit{„Joc secund”}, balada \operatitle\ face parte din a doua etapă de creație barbiană, numită baladic-orientală, dar anunță dezvoltarea ulterioară a poeziei lui \operaauthor.


\section{Evidențierea trăsăturilor care fac posibilă încadrarea poeziei studiate într-o tipologie, într-un curent cultural/literar, într-o orientare tematică}

\marginnote{poem alegoric}[1.2cm]
\operatitle\ este subintitulată \textit{„Baladă”}, începe ca un cântec bătrânesc de nuntă, dar se realizează în viziune modernă, ca un amplu poem de cunoaștere și poem alegoric, o poveste de iubire din lumea vegetală. Autorul păstrează din specia tradițională schema epică și personajele antagonice, dar evenimentele narate sunt de natură fantastică (dialogul în vis dintre rigă și laponă) și alegorică. Scenariul epic este dublat de caracterul dramatic și de \textit{„lirismul de măști”}, personajele având semnificații simbolice multiple (materia și spiritul etc.).

Poemul se încadrează modernismului interbelic prin intelectualizarea emoției, imaginar poetic inedit, ambiguitate, metafore surprinzătoare și cuvinte cu sonorități neobișnuite, înnoiri prozodice.


\section{Prezentarea imaginilor/ideilor poetice, relevante pentru te\-ma și viziunea despre lume din textul studiat}

Tema poeziei o reprezintă iubirea ca modalitate de cunoaștere a lumii. Fiind „un \textbfit{Luceafăr} întors”, poemul prezintă drama cunoașterii și a incompatibilității dintre două lumi (regnuri).

Titlul baladei trimite cu gândul la marile povești de dragoste din literatura universală, \textbfit{„Romeo și Julieta”}, \textbfit{„Tristan și Isolda”}. Însă la \operaauthor, membrii cuplului sunt antagonici (fac parte din regnuri diferite). Sunt personaje romantice cu trăsături excepționale, dar negative în raport cu norma comună (Crypto e \textit{„sterp”} și \textit{„nărăvaș/Că nu voia să înflorească”}, iar Enigel este \textit{„prea-cuminte”}).

\marginnote{semnificația titlului}[-0.3cm]
Numele Crypto are dublă semnificație: cel tăinuit, \textit{„inimă ascunsă”}, provenind din adjectivul \textit{„criptic”}, (\textit{„ascuns”}, \textit{„tăinuit”}), dar sugerează, în egală măsură, apartenența sa la familia ciupercilor, numele științific \textit{„criptogame”}. Personajul este rege (rigă) al făpturilor inferioare, din regnul vegetal. Numele cu sonoritate nordică Enigel sugerează originea laponei (de la pol) și trimite probabil la semnificația cuvântului din limba suedeză, \textit{„înger”} (care provine din latinescul \textit{„angelus”}).


\section{Ilustrarea elementelor de compoziție și de limbaj ale textului poetic studiat, semnificative pentru tema și viziunea despre lume {\footnotesize\normalfont(de exemplu: imaginar poetic, titlu, incipit, relații de opoziție și de simetrie, motiv poetic, laitmotiv, figuri semantice/tropi, elemente de prozodie etc.)}}

\marginnote{compoziție}[0.3cm]
La nivel formal, poezia este alcătuită din două părți, fiecare dintre ele prezentând câte o nuntă: una împlinită, cadru al celeilalte nunți, povestită, ratată, modificată în final prin căsătoria lui Crypto cu măsălarița. Formula compozițională este aceea a povestirii în ramă.

Prologul conturează în puține imagini atmosfera de la finalul unei nunți trăite. Primele patru strofe constituie rama viitoarei povești și reprezintă dialogul menestrelului cu \textit{„nuntașul fruntaș”}.

Partea a doua prezintă povestea de iubire neîmplinită dintre Enigel și riga Crypto. Nunta povestită cuprinde mai multe tablouri poetice: portretul și împărăția rigăi Crypto (strofele 5 -- 7), portretul, locurile natale și oprirea din drum a laponei Enigel (strofele 8 -- 9), întâlnirea dintre cei doi (strofa 10), cele trei chemări ale rigăi și primele două refuzuri ale laponei (strofele 11 -- 15), răspunsul laponei și refuzul categoric cu relevarea relației dintre simbolul solar și propria condiție (strofele 16 -- 20), încheierea întâlnirii (strofele 21 -- 22), pedepsirea rigăi în finalul baladei (strofele 23 -- 27). Modurile de expunere sunt, în ordine: descrierea, dialogul și narațiunea.

În expozițiune, sunt prezentate în antiteză portretele membrilor cuplului și locurile lor natale, deosebirile dintre ei generând intriga.

Riga Crypto, \textit{„inimă ascunsă”}, este craiul bureților, căruia dragostea pentru Enigel, \textit{„laponă mică, liniștită”}, îi este fatală. Singura lor asemănare este statutul superior în interiorul propriei lumi: el este rigă al plantelor inferioare, care nu înfloresc, iar păstorița care își conduce turmele de reni spre sud este o stăpână a regnului animal, în ipostaza de ființă rațională, omul -- \textit{„fiară bătrână”}.

\marginnote{imaginar poetic}[0.3cm]
Spațiul definitoriu al existenței, pentru Crypto, este umezeala perpetuă și impură, în timp ce lapona vine \textit{„din țări de gheață urgisită”}, spațiu rece, ceea ce explică aspirația ei spre soare și lumină, dar și mișcarea de transhumanță care ocazionează popasul în ținutul rigăi.

Membrii cuplului fac parte din regnuri diferite și, de aceea, nu pot comunica în plan real. Întâlnirea lor se realizează în visul fetei, la fel ca în \textbfit{„Luceafărul”}. Riga este cel care rostește de trei ori descântecul de dragoste și, de tot atâtea ori, lapona îl respinge. Povestea propriu-zisă se dovedește a fi fantastică, ca și în poemul eminescian, doar că rolurile sunt inversate. În dialogul lor, formulele de adresare sugerează familiaritate, afecțiune blândă: repetiția \textit{„Enigel, Enigel”}, epitetul \textit{„rigă blând”}.

În prima chemare-descântec, cu rezonanțe de incantație magică, Crypto își îmbie aleasa cu \textit{„dulceață”} și cu \textit{„fulgi”}, elemente ale existenței sale vegetative, dar care aici capătă conotații erotice. Darul lui este refuzat categoric de Enigel: \textit{„Eu mă duc să culeg/Fragii fragezi mai la vale”}. Refuzul laponei îl pune într-o situație dilematică, dar opțiunea lui e fermă și merge până la sacrificiul de sine, în a doua chemare: \textit{„Dacă pleci să culegi/Începi, rogu-te, cu mine”}.

Al doilea refuz este susținut de enumerarea atributelor lui Crypto: \textit{„blând”}, \textit{„plăpând”}, \textit{necopt -- „Lasă. Așteaptă de te coace”}. Opoziția \textit{„copt”} -- \textit{„necopt”}, reluată în al treilea refuz prin antiteza \textit{soare-umbră}, pune în evidență incompatibilitatea lor. Imaginii de fragilitate a lui Crypto lapona îi opune aspirația ei spre absolut. Soarele este simbolul existenței spirituale, al împlinirii umane, în antiteză cu \textit{„umbra”}, simbol al existenței instinctuale, sterile, vegetative.

Pentru a-și continua drumul către soare și cunoaștere, lapona refuză descântecul rigăi, deși regretă și plânge. Descântecul se întoarce în mod brutal asupra celui care l-a rostit și-l distruge. Făptura firavă este distrusă de propriul vis, cade victimă neputinței și îndrăznelii de a-și depăși limitele.

Finalul este trist. Riga Crypto se transformă într-o ciupercă otrăvitoare, obligat să rămână alături de făpturi asemenea lui, \textit{„Laurul-Balaurul”} și \textit{„măsălarița-mireasă”}. Încercarea ființei inferioare de a-și depăși limitele este pedepsită cu nebunia.

\marginnote{figuri semantice/ tropi}[1.3cm]
Soarele, simbolul spiritului, este imaginat în poem prin metaforele \textit{„roata albă”} (perfecțiunea geometrică) și \textit{„aprins inel”} (simbolul nunții), în antiteză cu \textit{„umbra”}, iar metafora \textit{„sufletul-fântână”} sugerează puritatea, setea de cunoaștere, veșnicia, fiind în antiteză cu \textit{„carnea”} (trupul, instinctele). Spiritul și sufletul sunt atribute ale ființei raționale, înțelepte. Făpturile inferioare care aspiră să dobândească spiritualitate sunt distruse de propriul vis, așa cum i se întâmplă lui Crypto, care „înnebunește” și se transformă în ciupercă otrăvitoare.

Trei mituri fundamentale de origine greacă sunt valorificate în opera poetului: al soarelui (absolutul), al nunții și al oglinzii.

Drumul spre sud al laponei are semnificația unui drum inițiatic, iar popasul în ținutul rigăi este o probă, trecută prin respingerea nunții pe o treaptă inferioară.
\marginnote{semnificații}[0.3cm]
Drumul trece prin cercul Venerii (iubirea ce reduce omul la ipostază de ființă instinctuală), apoi sufletul trebuie să mai urce o treaptă, cercul lui Mercur, mai pur, al intelectului, al cunoașterii raționale. Inițierea completă are loc prin adevărata \textit{nuntă} a trupului și spiritului cu însuși focarul vieții, prin trecerea omului în cercul Soarelui (cunoașterea absolută). Aspirația solară a laponei sugerează faptul că, în momentul întâlnirii cu riga Crypto, aceasta se află pe treapta lui Mercur, fără ca ea să fi trăit experiența iubirii. Chemările lui Crypto o atrag spre cercul Venerii. Ea trăiește iubirea ca experiență inițiatică, dar alege să-și urmeze drumul spre Soare (cunoașterea absolută).

\marginnote{figuri de stil}[0.8cm]
Sub raport stilistic, prezența inversiunilor (\textit{„zice-l-aș”}) și a vocativelor în prima parte a baladei evidențiază oralitatea textului. În portretizarea celor două personaje simbolice sunt utilizate epitetul și antiteza: Crypto este \textit{„sterp și nărăvaș”}, \textit{„rigă spân”}; lapona e \textit{„mică, liniștită”} și \textit{„prea-cuminte”}. Ambiguitatea este produsă de metaforele insolite: \textit{„Că sufletul nu e fântână/}[...]\textit{/Pahar e gândul, cu otravă”}.

\marginnote{prozodie}[0.3cm]
Alcătuirea prozodică pare destul de riguroasă inițial: catrene cu rimă încrucișată și măsură predominantă de 8 -- 9 silabe.


\subsection{Concluzie}

Poemul \operatitle\ impune o viziune modernă. Interpretarea dată de însuși \operaauthor\ poemului, \textit{„un \textbf{Luceafăr} întors”}, relevă asemănarea cu problematica capodoperei lui Mihai Eminescu, dar poemul modern este totuși \textit{„un Luceafăr cu rolurile inversate și într-un decor de o nebănuită noutate”}, cum remarcă Nicolae Manolescu.


\chapter{Eseu despre particularitățile de construcție a personajului principal dintr-o comedie studiată}
% Commands
\renewcommand{\operatitle}{\textbfit{„Riga Crypto și lapona Enigel”}} % title of the text
\renewcommand{\operaauthor}{Ion Barbu} % author of the text


% Beginning of text
\subsection{Context}

Publicată în 1924, integrată apoi în volumul \textbfit{„Joc secund”}, balada \operatitle\ face parte din a doua etapă de creație barbiană, numită baladic-orientală, dar anunță dezvoltarea ulterioară a poeziei lui \operaauthor.


\section{Evidențierea trăsăturilor care fac posibilă încadrarea poeziei studiate într-o tipologie, într-un curent cultural/literar, într-o orientare tematică}

\marginnote{poem alegoric}[1.2cm]
\operatitle\ este subintitulată \textit{„Baladă”}, începe ca un cântec bătrânesc de nuntă, dar se realizează în viziune modernă, ca un amplu poem de cunoaștere și poem alegoric, o poveste de iubire din lumea vegetală. Autorul păstrează din specia tradițională schema epică și personajele antagonice, dar evenimentele narate sunt de natură fantastică (dialogul în vis dintre rigă și laponă) și alegorică. Scenariul epic este dublat de caracterul dramatic și de \textit{„lirismul de măști”}, personajele având semnificații simbolice multiple (materia și spiritul etc.).

Poemul se încadrează modernismului interbelic prin intelectualizarea emoției, imaginar poetic inedit, ambiguitate, metafore surprinzătoare și cuvinte cu sonorități neobișnuite, înnoiri prozodice.


\section{Prezentarea imaginilor/ideilor poetice, relevante pentru te\-ma și viziunea despre lume din textul studiat}

Tema poeziei o reprezintă iubirea ca modalitate de cunoaștere a lumii. Fiind „un \textbfit{Luceafăr} întors”, poemul prezintă drama cunoașterii și a incompatibilității dintre două lumi (regnuri).

Titlul baladei trimite cu gândul la marile povești de dragoste din literatura universală, \textbfit{„Romeo și Julieta”}, \textbfit{„Tristan și Isolda”}. Însă la \operaauthor, membrii cuplului sunt antagonici (fac parte din regnuri diferite). Sunt personaje romantice cu trăsături excepționale, dar negative în raport cu norma comună (Crypto e \textit{„sterp”} și \textit{„nărăvaș/Că nu voia să înflorească”}, iar Enigel este \textit{„prea-cuminte”}).

\marginnote{semnificația titlului}[-0.3cm]
Numele Crypto are dublă semnificație: cel tăinuit, \textit{„inimă ascunsă”}, provenind din adjectivul \textit{„criptic”}, (\textit{„ascuns”}, \textit{„tăinuit”}), dar sugerează, în egală măsură, apartenența sa la familia ciupercilor, numele științific \textit{„criptogame”}. Personajul este rege (rigă) al făpturilor inferioare, din regnul vegetal. Numele cu sonoritate nordică Enigel sugerează originea laponei (de la pol) și trimite probabil la semnificația cuvântului din limba suedeză, \textit{„înger”} (care provine din latinescul \textit{„angelus”}).


\section{Ilustrarea elementelor de compoziție și de limbaj ale textului poetic studiat, semnificative pentru tema și viziunea despre lume {\footnotesize\normalfont(de exemplu: imaginar poetic, titlu, incipit, relații de opoziție și de simetrie, motiv poetic, laitmotiv, figuri semantice/tropi, elemente de prozodie etc.)}}

\marginnote{compoziție}[0.3cm]
La nivel formal, poezia este alcătuită din două părți, fiecare dintre ele prezentând câte o nuntă: una împlinită, cadru al celeilalte nunți, povestită, ratată, modificată în final prin căsătoria lui Crypto cu măsălarița. Formula compozițională este aceea a povestirii în ramă.

Prologul conturează în puține imagini atmosfera de la finalul unei nunți trăite. Primele patru strofe constituie rama viitoarei povești și reprezintă dialogul menestrelului cu \textit{„nuntașul fruntaș”}.

Partea a doua prezintă povestea de iubire neîmplinită dintre Enigel și riga Crypto. Nunta povestită cuprinde mai multe tablouri poetice: portretul și împărăția rigăi Crypto (strofele 5 -- 7), portretul, locurile natale și oprirea din drum a laponei Enigel (strofele 8 -- 9), întâlnirea dintre cei doi (strofa 10), cele trei chemări ale rigăi și primele două refuzuri ale laponei (strofele 11 -- 15), răspunsul laponei și refuzul categoric cu relevarea relației dintre simbolul solar și propria condiție (strofele 16 -- 20), încheierea întâlnirii (strofele 21 -- 22), pedepsirea rigăi în finalul baladei (strofele 23 -- 27). Modurile de expunere sunt, în ordine: descrierea, dialogul și narațiunea.

În expozițiune, sunt prezentate în antiteză portretele membrilor cuplului și locurile lor natale, deosebirile dintre ei generând intriga.

Riga Crypto, \textit{„inimă ascunsă”}, este craiul bureților, căruia dragostea pentru Enigel, \textit{„laponă mică, liniștită”}, îi este fatală. Singura lor asemănare este statutul superior în interiorul propriei lumi: el este rigă al plantelor inferioare, care nu înfloresc, iar păstorița care își conduce turmele de reni spre sud este o stăpână a regnului animal, în ipostaza de ființă rațională, omul -- \textit{„fiară bătrână”}.

\marginnote{imaginar poetic}[0.3cm]
Spațiul definitoriu al existenței, pentru Crypto, este umezeala perpetuă și impură, în timp ce lapona vine \textit{„din țări de gheață urgisită”}, spațiu rece, ceea ce explică aspirația ei spre soare și lumină, dar și mișcarea de transhumanță care ocazionează popasul în ținutul rigăi.

Membrii cuplului fac parte din regnuri diferite și, de aceea, nu pot comunica în plan real. Întâlnirea lor se realizează în visul fetei, la fel ca în \textbfit{„Luceafărul”}. Riga este cel care rostește de trei ori descântecul de dragoste și, de tot atâtea ori, lapona îl respinge. Povestea propriu-zisă se dovedește a fi fantastică, ca și în poemul eminescian, doar că rolurile sunt inversate. În dialogul lor, formulele de adresare sugerează familiaritate, afecțiune blândă: repetiția \textit{„Enigel, Enigel”}, epitetul \textit{„rigă blând”}.

În prima chemare-descântec, cu rezonanțe de incantație magică, Crypto își îmbie aleasa cu \textit{„dulceață”} și cu \textit{„fulgi”}, elemente ale existenței sale vegetative, dar care aici capătă conotații erotice. Darul lui este refuzat categoric de Enigel: \textit{„Eu mă duc să culeg/Fragii fragezi mai la vale”}. Refuzul laponei îl pune într-o situație dilematică, dar opțiunea lui e fermă și merge până la sacrificiul de sine, în a doua chemare: \textit{„Dacă pleci să culegi/Începi, rogu-te, cu mine”}.

Al doilea refuz este susținut de enumerarea atributelor lui Crypto: \textit{„blând”}, \textit{„plăpând”}, \textit{necopt -- „Lasă. Așteaptă de te coace”}. Opoziția \textit{„copt”} -- \textit{„necopt”}, reluată în al treilea refuz prin antiteza \textit{soare-umbră}, pune în evidență incompatibilitatea lor. Imaginii de fragilitate a lui Crypto lapona îi opune aspirația ei spre absolut. Soarele este simbolul existenței spirituale, al împlinirii umane, în antiteză cu \textit{„umbra”}, simbol al existenței instinctuale, sterile, vegetative.

Pentru a-și continua drumul către soare și cunoaștere, lapona refuză descântecul rigăi, deși regretă și plânge. Descântecul se întoarce în mod brutal asupra celui care l-a rostit și-l distruge. Făptura firavă este distrusă de propriul vis, cade victimă neputinței și îndrăznelii de a-și depăși limitele.

Finalul este trist. Riga Crypto se transformă într-o ciupercă otrăvitoare, obligat să rămână alături de făpturi asemenea lui, \textit{„Laurul-Balaurul”} și \textit{„măsălarița-mireasă”}. Încercarea ființei inferioare de a-și depăși limitele este pedepsită cu nebunia.

\marginnote{figuri semantice/ tropi}[1.3cm]
Soarele, simbolul spiritului, este imaginat în poem prin metaforele \textit{„roata albă”} (perfecțiunea geometrică) și \textit{„aprins inel”} (simbolul nunții), în antiteză cu \textit{„umbra”}, iar metafora \textit{„sufletul-fântână”} sugerează puritatea, setea de cunoaștere, veșnicia, fiind în antiteză cu \textit{„carnea”} (trupul, instinctele). Spiritul și sufletul sunt atribute ale ființei raționale, înțelepte. Făpturile inferioare care aspiră să dobândească spiritualitate sunt distruse de propriul vis, așa cum i se întâmplă lui Crypto, care „înnebunește” și se transformă în ciupercă otrăvitoare.

Trei mituri fundamentale de origine greacă sunt valorificate în opera poetului: al soarelui (absolutul), al nunții și al oglinzii.

Drumul spre sud al laponei are semnificația unui drum inițiatic, iar popasul în ținutul rigăi este o probă, trecută prin respingerea nunții pe o treaptă inferioară.
\marginnote{semnificații}[0.3cm]
Drumul trece prin cercul Venerii (iubirea ce reduce omul la ipostază de ființă instinctuală), apoi sufletul trebuie să mai urce o treaptă, cercul lui Mercur, mai pur, al intelectului, al cunoașterii raționale. Inițierea completă are loc prin adevărata \textit{nuntă} a trupului și spiritului cu însuși focarul vieții, prin trecerea omului în cercul Soarelui (cunoașterea absolută). Aspirația solară a laponei sugerează faptul că, în momentul întâlnirii cu riga Crypto, aceasta se află pe treapta lui Mercur, fără ca ea să fi trăit experiența iubirii. Chemările lui Crypto o atrag spre cercul Venerii. Ea trăiește iubirea ca experiență inițiatică, dar alege să-și urmeze drumul spre Soare (cunoașterea absolută).

\marginnote{figuri de stil}[0.8cm]
Sub raport stilistic, prezența inversiunilor (\textit{„zice-l-aș”}) și a vocativelor în prima parte a baladei evidențiază oralitatea textului. În portretizarea celor două personaje simbolice sunt utilizate epitetul și antiteza: Crypto este \textit{„sterp și nărăvaș”}, \textit{„rigă spân”}; lapona e \textit{„mică, liniștită”} și \textit{„prea-cuminte”}. Ambiguitatea este produsă de metaforele insolite: \textit{„Că sufletul nu e fântână/}[...]\textit{/Pahar e gândul, cu otravă”}.

\marginnote{prozodie}[0.3cm]
Alcătuirea prozodică pare destul de riguroasă inițial: catrene cu rimă încrucișată și măsură predominantă de 8 -- 9 silabe.


\subsection{Concluzie}

Poemul \operatitle\ impune o viziune modernă. Interpretarea dată de însuși \operaauthor\ poemului, \textit{„un \textbf{Luceafăr} întors”}, relevă asemănarea cu problematica capodoperei lui Mihai Eminescu, dar poemul modern este totuși \textit{„un Luceafăr cu rolurile inversate și într-un decor de o nebănuită noutate”}, cum remarcă Nicolae Manolescu.


\end{document}
