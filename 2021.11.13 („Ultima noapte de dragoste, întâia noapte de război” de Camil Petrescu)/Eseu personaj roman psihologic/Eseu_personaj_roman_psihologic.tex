%%
%% Basic LaTeX template by Andy3153
%% created   04/10/21 ~ 18:21:32
%% modified1 15/10/21 ~ 23:15:05
%% modified2 03/11/21 ~ 18:37:45
%%

% Document preamble
\documentclass[
12pt,                        % Font size
a4paper                      % Paper type
]{article}

% Packages
\usepackage[
margin=2.7cm,                % Margin size
marginparwidth=2cm,          % Margin note size
marginparsep=3mm             % Space between margin and text
]{geometry}
\usepackage[utf8]{inputenc}  % UTF-8 support
\usepackage[T1]{fontenc}     % Proper hyphenation
\usepackage[romanian]{babel} % Romanian characters support
\usepackage{indentfirst}     % Add paragraph indentation even after a section
\usepackage{marginnote}      % Notes on the margins of a document (more advanced \marginpar)
\usepackage{titlesec}        % Customize titles

% Custom titles, sections, subsections etc. format
\titleformat*{\section}{\large\bfseries}
\titleformat{\subsection}{\normalfont\normalfont\bfseries}{}{0pt}{}

% Page numbering
%\pagenumbering{gobble} % uncomment if you want to disable it

% Custom commands
% Format: \newcommand{\command}{action (add '\ ' or '{}' if it won't add a space properly)}
\newcommand{\rom}[1]{\uppercase\expandafter{\romannumeral #1\relax}} % Roman numerals
\newcommand{\operatitle}{\textbf{\textit{„Ultima noapte de dragoste, întâia noapte de război”\ }}} % title of the commented opera
\newcommand{\operaauthor}{Camil Petrescu\ } % author of the commented opera

% Basic document info
\title{Eseu despre particularitățile de construcție a personajului principal dintr-un roman psihologic}
\date{}   % Show no date in the title
\author{} % Empty author to not get a warn about missing author

\begin{document}
\maketitle % Show the title
%\reversemarginpar % put margin notes on left instead of on right


% Beginning of text

\subsection{Context}

\operaauthor ilustrează estetica autenticității în studiile teoretice și prin intermediul romanelor sale (\operatitle 1930; \textbf{\textit{„Patul lui Procust”}} 1933).

\section{Elemente de structură și de compoziție ale romanului, semnificative pentru realizarea personajului din romanul studiat {\footnotesize (precum: acțiunea, conflictul, relațiile temporale și spațiale, incipitul, finalul, tehnicile narative, perspectiva narativă, registrele stilistice, limbajul personajelor etc.)}}

\marginnote{incipit/ final deschis}[0.8cm]
Chiar dacă este vorba de un roman modern, în incipit sunt fixate cu precizie realistă coordonatele spațio-temporale: \textit{„În primăvara anului 1916,} [...] \textit{între Bușteni și Predeal.”} În schimb, finalul deschis lasă loc interpretărilor multiple, așa cum se întâmplă în general în proza de analiză psihologică. Astfel, Gheorghidiu, obosit să mai caute certitudini și să se mai îndoiască, se simte detașat de tot ceea ce îl legase de Ela, hotărăște să o părăsească și să-i lase \textit{„tot trecutul”}.

În romanul lui Camil Petrescu, apare conflictul interior, din conștiința personajului narator, Ștefan Gheorghidiu, care trăiește stări și sentimente contradictorii față de soția sa, Ela.
\marginnote{conflict}[0.3cm]
Acest conflict interior este generat de raporturile pe care protagonistul le are cu realitatea înconjurătoare. Principalul motiv al rupturii dintre Ștefan și soția sa este implicarea Elei în lumea mondenă, pe care eroul o disprețuiește. Așadar, conflictul interior se produce din cauza diferenței dintre aspirațiile lui Gheorghidiu și realitatea lumii înconjurătoare.

Conflictul exterior scoate în evidență relația personajului cu societatea, protagonistul fiind plasat în categoria inadaptaților social.

\marginnote{perspectiva narativă}[0cm]
Romanul este scris la persoana \rom{1}, sub forma unei confesiuni a personajului principal, Ștefan Gheorghidiu, care trece prin două experiențe fundamentale: iubirea și războiul. Relatarea la persoana întâi conferă autenticitate și caracter subiectiv textului.

\marginnote{structura}[0cm]
Romanul este alcătuit din două părți și treisprezece capitole cu titluri sugestive.

\clearpage

\section{Precizarea statutului social, psihologic, moral etc. al personajului ales}

Personajul principal Ștefan Gheorghidiu reprezintă tipul intelectualului lucid, inadaptabilul superior, care nu-și găsește locul într-o societate dominată de mediocritate și de lipsă de moralitate.

Student la Filosofie, Ștefan Gheorghidiu este un intelectual care trăiește în lumea ideilor, a cărților și care are impresia că s-a izolat de realitatea materială imediată. Însă tocmai această realitate imediată produce destrămarea cuplului pe care el îl formează cu Ela. Până în momentul în care Gheorghidiu primește moștenirea de la unchiul Tache, cuplul trăiește în condiții modeste, dar în armonie.

\section{Ilustrarea trăsăturilor personajului ales, prin secvențe na\-ra\-ti\-ve/si\-tua\-ții semnificative sau prin citate comentate}

\marginnote{orgoliul, trăsătură definitorie a personajului}[0.3cm]
Consider că principala trăsătură de caracter a protagonistului este orgoliul. Ilustrativă în acest sens este mărturisirea lui Gheorghidiu referitoare la felul în care ia naștere iubirea lui pentru Ela: \textit{„Iubești întâi din milă, din îndatorire, din duioșie, iubești pentru că știi că asta o face fericită”}; dar la o autoanaliză lucidă, naratorul recunoaște că: \textit{„Începusem totuși să fiu măgulit de admirația pe care o avea mai toată lumea pentru mine, fiindcă eram atât de pătimaș iubit de una dintre cele mai frumoase studente, și cred că acest orgoliu a constituit baza viitoarei mele iubiri.”}

O secvență narativă semnificativă pentru a ilustra orgoliul personajului este aceea a mesei în familie din casa bătrânului avar, Tache, prilej cu care acesta din urmă hotărăște să-i lase cea mai mare parte din avere lui Ștefan. Astfel, Ela și Ștefan sunt invitați la masă la unchiul Tache, unde celălalt unchi, Nae Gheorghidiu, ironizează căsătoria din dragoste cu o fată săracă, pe care i-o reproșează atât lui Ștefan, cât și tatălui său mort, Corneliu, pe care în plus îl acuză că nu -ia lăsat nicio moștenire fiului. Ștefan încearcă să-și apere tatăl și le spune unchilor lui ce crede despre ei.

În mod surprinzător, unchiul Tache este impresionat de izbucnirea de sinceritate a lui Ștefan și îi lasă acestuia din urmă cea mai însemnată parte a averii, deși Nae Gheorghidiu remarcă mai târziu că nepotul lui este lipsit de spirit practic.

Moștenirea va genera numeroase discuții familiale, fiindcă atât Nae Gheorghidiu, cât și mama și surorile lui Ștefan îi vor intenta proces pentru a obține o parte cât mai mare din avere. Atitudinea soției care se implică cu îndârjire în discuțiile despre bani îl surprinde în mod dureros: \textit{„Aș fi vrut-o mereu feminină, deasupra discuțiilor acestea vulgare, plăpândă și având nevoie să fie protejată, nu să intervină atât de energic interesată”}. Primirea moștenirii generează criza matrimonială, fiindcă se pare că Ela se lasă în voia tentațiilor mondene și începe să-și compare soțul cu dansatorii din noul lor cerc de prieteni, în dezavantajul lui Ștefan.

Tot din orgoliu, el refuză să intre în competiție cu ceilalți, fiindcă i se pare sub demnitatea lui de intelectual să-și schimbe garderoba și să adopte comportamentul superficial al dansatorilor mondeni apreciați de Ela.

Din dorința de a trăi o experiență existențială pe care o consideră definitorie pentru formarea lui ca om, dar și din orgoliu, Ștefan se înrolează voluntar, deși ar fi putut să evite participarea la război, folosindu-se de averea sa, așa cum procedează Nae Gheorghidiu sau cumnatul său, Iorgu.

Pe lângă orgoliu, în roman se conturează și alte trăsături ale eroului, precum: natura analitică și reflexivă, luciditatea, sensibilitatea exagerată, inteligența, conștiința propriei valori în raport cu filfizonii mondeni apreciați de Ela.

În ultimul capitol, deznodământul înfățișează efectele celor două experiențe asupra personajului. Face trimitere la o scrisoare anonimă pe care o primește Gheorghidiu și în care i se dezvăluie că soția îl înșală. Ștefan nu mai este însă interesat să verifice autenticitatea acestei scrisori. Bărbatul care odinioară se credea capabil de crimă din gelozie devine indiferent și detașat de iubirea din trecut, hotărând să-i lase soției \textit{„tot trecutul.”}

\marginnote{in\-tro\-spec\-ția și monolog interior}[0.3cm]
Prin introspecție și monolog interior, tehnici ale analizei psihologice, Ștefan Gheorghidiu își analizează cu luciditate trăirile, stările și sentimentele.

Portretul lui Gheorghidiu este realizat mai ales prin caracterizare indirectă, care se desprinde din fapte, gânduri, limbaj, gesturi, atitudini și din relațiile cu celelalte personaje. Caracterizarea directă este realizată rar, aceea pe care i-o adresează Ela lui Ștefan când soțul îi reproșează comportamentul ei din timpul excursiei de la Odobești: \textit{„Ești de o sensibilitate imposibilă.”}

Este folosită adesea autocaracterizarea, pentru portretul fizic, moral sau psihologic (\textit{„Eram înalt și elegant”}).

\marginnote{autoanaliză lucidă}[0.3cm]
Pentru observarea propriilor trăiri se utilizează procedee specifice romanului psihologic modern precum: introspecția (\textit{„Niciodată nu m-am simțit mai descheiat de mine însumi, mai nenorocit.”}), autoanaliza lucidă (\textit{„Simțeam din zi în zi, departe de  femeia mea, că voi muri.”}), monologul interior, cu notarea stărilor fiziologice și a senzațiilor organice (\textit{„Nu pot gândi nimic. Creierul parcă mi s-a zemuit, nervii, de atâta încordare, s-au rupt ca niște sfori putrede”}), fluxul conștiinței.

\subsection{Concluzie}

\operatitle este un roman modern, psihologic, de tip subiectiv, reprezentativ pentru o nouă viziune autentică și demitizată asupra războiului. Prin Ștefan Gheorghidiu, personajul-narator, scriitorul impune în literatura română o nouă tipologie: intelectualul inadaptat, aspirând spre absolut.
\end{document}
